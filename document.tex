%%This is a very basic article template.
%%There is just one section and two subsections.
\documentclass[a4j,11pt]{jarticle}
\usepackage{ascmac}
\usepackage{listings}
\usepackage[dvips]{graphicx}
\usepackage{eclclass}
\usepackage{listings,jlisting}
\usepackage{setspace}
\setstretch{1}
%%ローマ数字に変換
%###
\def\rnum#1{\expandafter{\romannumeral #1}} 
\def\Rnum#1{\uppercase\expandafter{\romannumeral #1}} 
\def\hado{$\downarrow$ $\searrow$ $\rightarrow$}%%波動コマンド
\def\tatsu{$\downarrow$ $\swarrow$ $\leftarrow$}%%竜巻コマンド
\def\syoryu{$\rightarrow$ $\downarrow$ $\searrow$}%%昇竜コマンド
\def\gyakusyoryu{\leftarrow$ $\downarrow$ $\swarrow$}%%逆昇竜コマンド
\def\yoga{$\leftarrow$ $\swarrow$ $\downarrow$ $\searrow$ $\rightarrow$}%%ヨガフレ
\def\gyakuyoga{$\rightarrow$ $\searrow$ $\downarrow$ $\swarrow$ $\leftarrow$}%%逆ヨガ
\def\tenti{$\rightarrow$ $\searrow$ $\downarrow$ $\swarrow$ $\leftarrow$ $\rightarrow$}%%天地
\def\ryuko{$\downarrow$ $\searrow$ $\rightarrow$ $\searrow$ $\downarrow$ $\swarrow$ $\leftarrow$}%%龍虎
\def\orochi{$\downarrow$ $\swarrow$ $\leftarrow$ $\swarrow$ $\downarrow$ $\searrow$ $\rightarrow$}%%大蛇薙
\def\ue{$uparrow$}
\def\migiue{$nearrow$}
\def\hidariue{$nwarrow$}
\def\sita{$downarrow$}
\def\hidarisita{$swarrow$}
\def\migisita{$searrow$}
\def\migi{$rightarrow$}
\def\hidari{$leftarrow$}
\def\Cancel{$\Longrightarrow$}
\def\DC{DC$\Rightarrow$}
\def\SC{SC$\Rightarrow$}
\def\HC{HC$\Rightarrow$}
\def\MAXC{MAX$\Rightarrow$}
%%###
%%タイトルとかの設定
\title{KOF13コンボ一覧}
\author{}

\begin{document}
\maketitle
\tableofcontents % 目次が出現
\newpage
\part{概要}
\section{このテキストについて}
このテキストは、KOF13CLIMAX・家庭用・Steam Editionのコンボのほか、覚えておくと得するかもしれない程度の情報が記載されています。
コンボの難易度も5段階で表記していますし、ダメージ・ゲージ使用量も明記してあるので、KOF13を始めた方は是非。
\section{表記について}
ボタンと表記の対応は以下のとおりです。

\begin{screen}

 弱パンチ$\rightarrow$ $A$\ \ \ 強パンチ$\rightarrow$ $C$
 
 弱キック$\rightarrow$ $B$\ \ \ 強キック$\rightarrow$ $D$
\end{screen}
\vspace{11pt}

コンボの繋ぎについては以下のとおりに表記します。
冗長さを回避するため、技名は表記せず、コマンドのみを表記します。
\begin{screen}
\begin{tabular}{ll}
 キャンセル無し&\ \ \lbrack\ \ コマンド\ \ \rbrack\ \ $\longrightarrow$\ \ \lbrack\ \ コマンド\ \ \rbrack\ \ \\
 キャンセルする&\ \ \lbrack\ \ コマンド\ \ \rbrack\ \ $\Longrightarrow$\ \ \lbrack\ \ コマンド\ \ \rbrack\ \ \\
 ドライブキャンセル&\ \ \lbrack\ \ コマンド\ \ \rbrack\ \ DC$\Rightarrow$\ \ \lbrack\ \ コマンド\ \ \rbrack\ \ \\
 スーパーキャンセル&\ \ \lbrack\ \ コマンド\ \ \rbrack\ \ SC$\Rightarrow$\ \ \lbrack\ \ コマンド\ \ \rbrack\ \ \\
 ハイパードライブキャンセル&\ \ \lbrack\ \ コマンド\ \ \rbrack\ \ HC$\Rightarrow$\ \ \lbrack\ \ コマンド\ \ \rbrack\ \ \\
 MAXキャンセル&\ \ \lbrack\ \ コマンド\ \ \rbrack\ \ MAX$\Rightarrow$\ \ \lbrack\ \ コマンド\ \ \rbrack\ \ \\
 $n$段目キャンセル&\ \ \lbrack\ \ コマンド\ \ \rbrack\ \ $(n)$$\Rightarrow$\ \ \lbrack\ \ コマンド\ \ \rbrack\ \ 
\end{tabular}
\end{screen}

コマンドは、各チームの最初に記載。
\begin{itembox}[l]{注意}
\begin{itemize}
\item コマンドについては通常投げは記載しません。コマンドは右向き時のものです。
\item 前後の技との兼ね合いから、正規のコマンドで表記されていない技も有りますが、表記の通り入力すればコンボが成立します。
\item コマンドに$(\downarrow$ $\swarrow)$ $\leftarrow$ $\swarrow$ $\downarrow$ $\searrow$ $\rightarrow$ のように()がある場合、その部分を省略します。
\end{itemize}
\end{itembox}


\newpage
\part{日本チーム}
\section{コマンド}
\subsection{表京}
%%特殊技
\begin{itembox}[l]{特殊技}
\begin{tabular}{lll}
外式・轟斧\ 陽&$\rightarrow$\ +\ $B$&中段\\
八拾八式&$\searrow$\ +\ $D$&下段・二段技・初段キャンセル可\\
外式・奈落落とし&(空中で)$\swarrow$\ or\ $\downarrow$\ or\ $\searrow$\ +\ $C$&バクステ中も可・中段
\end{tabular}
\end{itembox}
%%必殺技
\begin{itembox}[l]{必殺技}
\begin{tabular}{lll}
百式・鬼焼き&$\rightarrow$ $\downarrow$ $\searrow$\ +\ $A$\ or\ $C$&弱上半身無敵・強全身無敵\\
百八式・闇払い&$\downarrow$ $\searrow$ $\rightarrow$\ +\ $A$\ or\ $C$&飛び道具判定・EX版弾貫通\\ 
百壱式・朧車&$\leftarrow$ $\downarrow$ $\swarrow$\ +\ $B$\ or\ $D$&EX版全身無敵\\
弐百拾弐式・琴月\ 陽&$\rightarrow$ $\searrow$ $\downarrow$ $\swarrow$ $\leftarrow$\ +\ $B$\ or\ $D$&EX版コマ投げ\\
七拾五式・改&$\downarrow$ $\searrow$ $\rightarrow$\ +\ $B$\ or\ $D$&二段技・連続ガードではない
\end{tabular}
\end{itembox}
%%超必殺技
\begin{itembox}[l]{超必殺技}
\begin{tabular}{lll}
裏百八式・大蛇薙&$\downarrow$ $\swarrow$ $\leftarrow$ $\swarrow$ $\downarrow$ $\searrow$ $\rightarrow$\ +\ $A$\ or\ $C$& 空中可・弾貫通・溜め可
\end{tabular}
\end{itembox}
%%NEOMAX超必殺技
\begin{itembox}[l]{NEOMAX超必殺技}
\begin{tabular}{lll}
裏弐百拾壱式・天叢雲&$\downarrow$ $\searrow$ $\rightarrow$ $\downarrow$ $\searrow$ $\rightarrow$\ +\ $A\ C$&弾貫通・ヒット時のみ演出
\end{tabular}
\end{itembox}
\newpage
\subsection{紅丸}
%%特殊技
\begin{itembox}[l]{特殊技}
\begin{tabular}{lll}
ジャックナイフキック&$\rightarrow$ \ +\ $B$&下段無敵\\
フライングドリル&(空中で)$\downarrow$\ +\ $D$&バクステからは不可
\end{tabular}
\end{itembox}
%%必殺技
\begin{itembox}[l]{必殺技}
\begin{tabular}{lll}
居合い蹴り&$\downarrow$ $\searrow$ $\rightarrow$ \ +\ $B$\ or\ $D$&GCABから確定反撃有り\\
反動三段蹴り&(居合い蹴り中)$\downarrow$ $\uparrow$\ +\ $B$\ or\ $D$&ディレイ可\\
スーパー稲妻キック&$\rightarrow$ $\downarrow$ $\searrow$\ +\ $B$\ or\ $D$&弱上半身無敵?・強完全無敵\\
紅丸ランサー&$\downarrow$ $\swarrow$ $\leftarrow$\ +\ $A$\ or\ $C$&EX版相手をサーチ\\
雷靭拳&$\downarrow$ $\searrow$ $\rightarrow$ \ +\ $A$\ or\ $C$&空中可\\
紅丸コレダー&$\rightarrow$ $\searrow$ $\downarrow$ $\swarrow$ $\leftarrow$ $\rightarrow$\ +\ $A$\ or\ $C$&1F投げ
\end{tabular}
\end{itembox}
%%超必殺技
\begin{itembox}[l]{超必殺技}
\begin{tabular}{lll}
雷光拳&$\downarrow$ $\searrow$ $\rightarrow$ $\downarrow$ $\searrow$ $\rightarrow$\ +\ $A$\ or\ $C$&EX全身無敵\\
紅丸ローリングサンダー&$\downarrow$ $\swarrow$ $\leftarrow$ $\downarrow$ $\swarrow$ $\leftarrow$\ +\ $A$\ or\ $C$&EX無し・全身無敵
\end{tabular}
\end{itembox}
%%NEOMAX超必殺技
\begin{itembox}[l]{NEOMAX超必殺技}
\begin{tabular}{lll}
雷皇拳&$\downarrow$ $\searrow$ $\rightarrow$ $\downarrow$ $\searrow$ $\rightarrow$\ +\ $B\ D$&方向キーで多少軌道が変化
\end{tabular}
\end{itembox}
\newpage
\subsection{大門}
%%特殊技
\begin{itembox}[l]{特殊技}
\begin{tabular}{lll}
頭上払い&$\searrow$\ +\ $C$&
\end{tabular}
\end{itembox}
%%必殺技
\begin{itembox}[l]{必殺技}
\begin{tabular}{lll}
天地返し&$\rightarrow$ $\searrow$ $\downarrow$ $\swarrow$ $\leftarrow$ $\rightarrow$\ +\ $A$\ or\ $C$&1F投げ\\
雲掴み投げ&$\leftarrow$ $\swarrow$ $\downarrow$ $\searrow$ $\rightarrow$\ +\ $A$&EX版無敵有り\\
切り株返し&$\leftarrow$ $\swarrow$ $\downarrow$ $\searrow$ $\rightarrow$\ +\ $C$&EX版無し・ダウン追い打ち\\
地雷震&$\rightarrow$ $\downarrow$ $\searrow$\ +\ $A$&地震判定・EX版初段中段\\
地雷震(フェイント)&$\rightarrow$ $\downarrow$ $\searrow$\ +\ $C$&EX版無し・攻撃判定無し\\
超受け身&$\downarrow$ $\swarrow$ $\leftarrow$\ +\ $B$\ or\ $D$&移動技・強/EX打撃無敵\\
超大外刈り&$\rightarrow$ $\downarrow$ $\searrow$\ +\ $B$\ or\ $D$&投げ技・発生まで無敵有り
\end{tabular}
\end{itembox}
%%超必殺技
\begin{itembox}[l]{超必殺技}
\begin{tabular}{lll}
地獄極楽落とし&$\rightarrow$ $\searrow$ $\downarrow$ $\swarrow$ $\leftarrow$ $\rightarrow$ $\searrow$ $\downarrow$ $\swarrow$ $\leftarrow$\ +\ $A$\ or\ $C$&1F投げ
\end{tabular}
\end{itembox}
%%NEOMAX超必殺技
\begin{itembox}[l]{NEOMAX超必殺技}
\begin{tabular}{lll}
驚天動地&$\downarrow$ $\searrow$ $\rightarrow$ $\downarrow$ $\searrow$ $\rightarrow$\ +\ $A\ C$&当身技・MAXキャンセル時打撃技
\end{tabular}
\end{itembox}
\newpage
\section{コンボ}
\subsection{表京}
コンボ始動は以下のように書き換えます。
\begin{itembox}[l]{基本コンボ始動技}
\begin{tabular}{ll}
\ \ \lbrack\ \ 近$C$\ \ \rbrack\ \ $\rightarrow$\ \ \lbrack\ \ $\searrow$ $D$\ \ \rbrack\ \ $(1)$&\rightarrow (\rnum{1})\\
\ \ \lbrack\ \ $\downarrow$ $B$$\ \ \rbrack\ \ $ $\rightarrow$$\ \ \lbrack\ \ $ $\downarrow$ $A$$\ \ \rbrack\ \ $ $\rightarrow$ $\ \ \lbrack\ \ $ $\searrow$ $D$$\ \ \rbrack\ \ $ $(1)$&$\rightarrow$ (\rnum{2})\\
\ \ \lbrack\ \ $\downarrow$ $B$\ \ \rbrack\ \ $\times$ 1~2$\rightarrow$\ \ \lbrack\ \ 立$B$\ \ \rbrack\ \ &\rightarrow (\rnum{3)}
\end{tabular}
\end{itembox}
\subsubsection{発動無し中央コンボ}
\begin{classify}{ (\rnum{1})\ or\  (\rnum{2})}
	\class{
		\begin{classify}{\Longrightarrow\ \ \lbrack\ \ $\downarrow$ $\searrow$ $\rightarrow$\ +\ $D$\ \ \rbrack\ \ }
			\class{
				\hspace{-12pt}
				\begin{tabular}[t]{lllll}
					\multicolumn{5}{l}{
						$\rightarrow$\ \ \lbrack\ \ \rightarrow$ $\searrow$ $\downarrow$ $\swarrow$ $\leftarrow$\ +\ $B$\ or\ $D$\ \ \rbrack\ \ 
						%%強75式→琴月
					}\\
					★☆☆☆☆&0PG&0\%&Damage&Stan\\
					\multicolumn{5}{l}{
						とりあえず安定。
					}\\
					\multicolumn{5}{l}{
						ゲージ回収力・ダメージともに高く、強制ダウンも取れる。
					}
				\end{tabular}
			}
			\class{
				\hspace{-12pt}
				\begin{tabular}[t]{lllll}
					\multicolumn{5}{l}{
						$\rightarrow$\ \ \lbrack\ \ \leftarrow$ $\downarrow$ $\swarrow$\ +\ $D$\ \ \rbrack\ \ 
						%%強75式→強朧
					}\\
					★☆☆☆☆&0PG&0\%&Damage&Stan\\
					\multicolumn{5}{l}{
						スタン値重視・運びもそこそこで強制ダウンも取れる。
					}
				\end{tabular}
			}
			\class{
				\begin{classify}{
					\hspace{-12pt}
					\begin{tabular}[t]{lllll}
						\multicolumn{5}{l}{
							$\rightarrow$\ \ \lbrack\ \ $\leftarrow$ $\downarrow$ $\swarrow$\ +\ $B$\ \ \rbrack\ \ 
							%%強75式→弱朧
						}
					\end{tabular}
				}
				\class{
					\hspace{-12pt}
					\begin{tabular}[t]{lllll}
						\multicolumn{5}{l}{
							$\rightarrow$ \ \ \lbrack\ \ $\leftarrow$ $\downarrow$ $\swarrow$\ +\ $B\ D$\ \ \rbrack\ \ 
							%%強75式→弱朧→EX朧
						}\\
						★☆☆☆☆&1PG&0\%&Damage&Stan\\
						\multicolumn{5}{l}{
							ほぼ端から端まで運べるルート。強制ダウン。
						}
					\end{tabular}
				}
				\class{
					\hspace{-12pt}
					\begin{tabular}[t]{lllll}
						\multicolumn{5}{l}{
							$\rightarrow$ \ \ \lbrack\ \ $\leftarrow$ $\downarrow$ $\swarrow$\ +\ $B$\ \ \rbrack\ \ $\rightarrow$ \ \ \lbrack\ \ $\downarrow$ $\swarrow$ $\leftarrow$ $\swarrow$ $\downarrow$ $\searrow$ $\rightarrow$\ +\ $A\ C$(微タメ)\ \ \rbrack\ \ 
							%%強75式→弱朧×2→EX大蛇薙
						}\\
						★☆☆☆☆&2PG&0\%&Damage&Stan\\
						\multicolumn{5}{l}{
							ちょいタメをすると発生が高速化することを利用したルート。
						}\\
						\multicolumn{5}{l}{
							相手を倒しきれる時に。
						}
					\end{tabular}
				}
				\class{
					\hspace{-12pt}
					\begin{tabular}[t]{lllll}
						\multicolumn{5}{l}{
							$\rightarrow$ \ \ \lbrack\ \ $\downarrow$ $\swarrow$ $\leftarrow$ $\swarrow$ $\downarrow$ $\searrow$ $\rightarrow$\ +\ $A\ C$\ \ \rbrack\ \ 
							%%強75式→弱朧→EX大蛇薙
						}\\
						★☆☆☆☆&2PG&0\%&Damage&Stan\\
						\multicolumn{5}{l}{
							ちょいタメが難しい人向け。
						}
					\end{tabular}
				}
				\end{classify}
			}
			%%一つのクラス
			\class{
				\hspace{-12pt}
				\begin{tabular}[t]{lllll}
					\multicolumn{5}{l}{
						$\rightarrow$\ \ \lbrack\ \ $\downarrow$ $\swarrow$ $\leftarrow$ $\swarrow$ $\downarrow$ $\searrow$ $\rightarrow$\ +\ $A$\ or\ $C$\ \ \rbrack\ \ 
						%%強75式→大蛇薙
					}\\
					★☆☆☆☆&1PG&0\%&Damage&Stan\\
					\multicolumn{5}{l}{
						超必殺技を絡めるルート。
					}\\
					\multicolumn{5}{l}{
						逆昇竜二回が難しい人はこちら。タメはしない
					}
				\end{tabular}
			}
			%%一つのクラスここまで
		\end{classify}
	}
\end{classify}
\newpage
\begin{classify}{(\rnum{3})}
	\class{
		\begin{classify}{
				\hspace{-12pt}
					\begin{tabular}[t]{lllll}
						\multicolumn{5}{l}{
							$\rightarrow$\ \ \lbrack\ \ $\downarrow$ $\searrow$ $\rightarrow$\ +\ $B$\ \ \rbrack\ \ 
							%%弱75式
						}
					\end{tabular}
		}
			%%一つのクラス
			\class{
				\hspace{-12pt}
				\begin{tabular}[t]{lllll}
					\multicolumn{5}{l}{
						$\rightarrow$\ \ \lbrack\ \ $\downarrow$ $\swarrow$ $\leftarrow$ $\swarrow$ $\downarrow$ $\searrow$ $\rightarrow$\ +\ $A$\ or\ $C(微タメ)\ \ \rbrack\ \ 
						%%弱75式→大蛇薙
					}\\
					★☆☆☆☆&1PG&0\%&Damage&Stan\\
					\multicolumn{5}{l}{
						猶予がないので難易度は高い。
					}\
				\end{tabular}
			}
			%%一つのクラスここまで
			%%一つのクラス
			\class{
				\hspace{-12pt}
				\begin{tabular}[t]{lllll}
					\multicolumn{5}{l}{
						$\rightarrow$\ \ \lbrack\ \ $\leftarrow$ $\downarrow$ $\swarrow$\ +\ $B$\ \ \rbrack\ \ $\rightarrow$ \ \ \lbrack\ \ $\downarrow$ $\swarrow$ $\leftarrow$ $\swarrow$ $\downarrow$ $\searrow$ $\rightarrow$\ +\ $A\ C$(微タメ)\ \ \rbrack\ \ 
						%%弱75式→大蛇薙
					}\\
					☆☆☆☆☆&2PG&0\%&Damage&Stan\\	%%データ
					\multicolumn{5}{l}{
						ちょいタメする必要があるが、通常大蛇薙よりは安定する。%%説明
					}\\
					\multicolumn{5}{l}{
						倒しきりたい時などに%%説明
					}
				\end{tabular}
			}
%%一つのクラスここまで
		\end{classify}
	}
	\class{
			\hspace{-12pt}
			\begin{tabular}[t]{lllll}
				\multicolumn{5}{l}{
					$\rightarrow$\ \ \lbrack\ \ \rightarrow$ $\searrow$ $\downarrow$ $\swarrow$ $\leftarrow$\ +\ $B$\ or\ $D$\ \ \rbrack\ \ 
					%%琴月
				}\\
				★☆☆☆☆&0PG&0\%&Damage&Stan\\
				\multicolumn{5}{l}{
					キャンセルが遅いとつながらない場合もあるが、簡単。
				}
			\end{tabular}
	}
\end{classify}
\subsubsection{発動無し端コンボ}
\begin{classify}{(\rnum{1})\ or\  (\rnum{2})}
%%一つのクラス
	\class{
		\begin{classify}{
			\hspace{-12pt}
			\begin{tabular}[t]{lllll}
				\multicolumn{5}{l}{
					\Longrightarrow\ \ \lbrack\ \ $\downarrow$ $\searrow$ $\rightarrow$\ +\ $D$\ \ \rbrack\ \ 
					$\rightarrow$\ \ \lbrack\ \ $\leftarrow$ $\downarrow$ $\swarrow$ \ +\ $B$\ \ \rbrack\ \ 
					%%強75式→弱朧
				}
			\end{tabular}
			}
			%%一つのクラス
			\class{
				\begin{classify}{
					\hspace{-12pt}
					\begin{tabular}[t]{lllll}
						\multicolumn{5}{l}{
							$\rightarrow$\ \ \lbrack\ \ $\downarrow$ $\searrow$ $\rightarrow$\ +\ $B$\ \ \rbrack\ \ 
							$\rightarrow$\ \ \lbrack\ \ $\leftarrow$ $\downarrow$ $\swarrow$\ +\ $B$\ \ \rbrack\ \ 
							$\rightarrow$\ \ \lbrack\ \ \rightarrow$ $\downarrow$ $\searrow$\ +\ $A$\ \ \rbrack\ \ %%コマンド一式
							%%強75式→弱朧→弱75→弱朧→弱鬼
						}\\
						☆☆☆☆☆&0PG&0\%&Damage&Stan\\	%%データ
						\multicolumn{5}{l}{
							端でノーゲージならこれ。ほぼ1ゲージ回収する。
						}
					\end{tabular}
				}
					%%一つのクラス
					\class{
						\begin{classify}{
							\hspace{-12pt}
							\begin{tabular}[t]{lllll}
								\multicolumn{5}{l}{
									\SC \ \lbrack\ \ \orochi \ +\ $A\ C$\ \ \rbrack\ \ %%コマンド一式
									$\rightarrow$\ \ \lbrack\ \ \hado \ +\ $A$\ \ \rbrack
									%%強75式→弱朧→弱75→弱朧→弱鬼SC空中EX大蛇薙→弱闇
								}\\
								\multicolumn{5}{l}{
									弱鬼焼きの二段目を早めにキャンセルする。ちょっと忙しい。
								}
							\end{tabular}
						}
							%%一つのクラス
							\class{
								\hspace{-12pt}
								\begin{tabular}[t]{lllll}
									\multicolumn{5}{l}{
										$\rightarrow$\ \ \lbrack\ \ \gyakuyoga \ +\ $B$\ or\ $D$\ \ \rbrack\ \ %%コマンド一式
										%%強75式→弱朧→弱75→弱朧→弱鬼SC空中EX大蛇薙→弱闇→琴月
									}\\
									☆☆☆☆☆&2PG&50\%&Damage&Stan\\	%%データ
									\multicolumn{5}{l}{
										ダメージ重視。こちらのほうがより早く動き始められる%%説明
									}
								\end{tabular}
							}
							%%一つのクラスここまで
							%%一つのクラス
							\class{
								\hspace{-12pt}
								\begin{tabular}[t]{lllll}
									\multicolumn{5}{l}{
										$\rightarrow$\ \ \lbrack\ \ \gyakusyoryu \ +\ $D$\ \ \rbrack\ \ %%コマンド一式
										%%強75式→弱朧→弱75→弱朧→弱鬼SC空中EX大蛇薙→弱闇→強朧
									}\\
									☆☆☆☆☆&2PG&50\%&Damage&Stan\\	%%データ
									\multicolumn{5}{l}{
										スタン値重視%%説明
									}
								\end{tabular}
							}
							%%一つのクラスここまで
						\end{classify}
					}
					%%一つのクラスここまで
				\end{classify}
			}
			%%一つのクラスここまで
			%%一つのクラス
			\class{
				\begin{classify}{
				\hspace{-12pt}
					\begin{tabular}[t]{lllll}
						\multicolumn{5}{l}{
							SC$\Rightarrow$\ \ \lbrack\ \ $(\downarrow$ $\swarrow)$ $\leftarrow$ $\swarrow$ $\downarrow$ $\searrow$ $\rightarrow$\ +\ $A$\ or\ $C$\ \ \rbrack\ \ %%コマンド一式
							$\rightarrow$\ \ \lbrack\ \ $\downarrow$ $\searrow$ $\rightarrow$\ +\ $A$\ \ \rbrack\ \ 
							%%強75式→弱朧DC→空中大蛇薙→弱闇払い
						}\\
						\multicolumn{5}{l}{
							前の弱朧で$\downarrow$ $\swarrow$までは入力されているのでこの入力でOK。
						}\\
					\end{tabular}
				}
					%%一つのクラス
					\class{
						\hspace{-12pt}
						\begin{tabular}[t]{lllll}
							\multicolumn{5}{l}{
								$\rightarrow$\ \ \lbrack\ \ $\rightarrow$ $\searrow$ $\downarrow$ $\swarrow$ $\leftarrow$\ +\ $B$\ or\ $D$\ \ \rbrack\ \ %%コマンド一式
								%%強75式→弱朧DC→空中大蛇薙→弱闇払い→琴月
							}\\
							☆☆☆☆☆&1PG&50\%&Damage&Stan\\	%%データ
							\multicolumn{5}{l}{
								ダメージ重視。こちらのほうがより早く動き始められる%%説明
							}
						\end{tabular}
					}
					%%一つのクラスここまで
					%%一つのクラス
					\class{
						\hspace{-12pt}
						\begin{tabular}[t]{lllll}
							\multicolumn{5}{l}{
								$\rightarrow$\ \ \lbrack\ \ \leftarrow$ $\downarrow$ $\swarrow$\ +\ $D$\ \ \rbrack\ \ %%コマンド一式
								%%強75式→弱朧DC→空中大蛇薙→弱闇払い→強朧
							}\\
							☆☆☆☆☆&1PG&50\%&Damage&Stan\\	%%データ
							\multicolumn{5}{l}{
								スタン値重視。%%説明
							}
						\end{tabular}
					}
					%%一つのクラスここまで
				\end{classify}
			}
			%%一つのクラスここまで
		\end{classify}
	}
%%一つのクラスここまで
\end{classify}
\newpage
\subsection{紅丸}
\subsection{大門}
\end{document}
