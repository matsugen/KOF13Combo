%%This is a very basic article template.
%%There is just one section and two subsections.
\documentclass[a4j,11pt]{jarticle}
\usepackage{ascmac}
\usepackage{listings}
\usepackage[dvips]{graphicx}
\usepackage{eclclass}
\usepackage{listings,jlisting}
\usepackage{setspace}
\usepackage{color}
\setstretch{1}
\pagestyle{empty}
\usepackage{theorem}
%%色作成
\definecolor{DG}{cmyk}{0.64,0,0.95,0.4}
\definecolor{PG}{cmyk}{0,0.7,1,0}
\definecolor{Acolor}{cmyk}{0,1,0.5,0.2}
\definecolor{Ccolor}{cmyk}{0,1,1,0.2}
\definecolor{Bcolor}{cmyk}{1,0.5,0,0.2}
\definecolor{Dcolor}{cmyk}{1,1,0,0.2}
%%ゲージやボタンのマクロ
\def\A{$\textcolor{Acolor}{A}$}
\def\C{$\textcolor{Ccolor}{C}$}
\def\B{$\textcolor{Bcolor}{B}$}
\def\D{$\textcolor{Dcolor}{D}$}
\def\PG#1{\textcolor{PG}{パワーゲージ\ :\ #1本}}
\def\DG#1{\textcolor{DG}{ドライブゲージ\ :\ #1\%}}
%%ローマ数字に変換
%###
\def\rnum#1{\expandafter{\romannumeral #1}} 
\def\Rnum#1{\uppercase\expandafter{\romannumeral #1}} 
\def\vtame{$\downarrow$\ タメ\ $\uparrow$}
\def\htame{$\leftarrow$\ タメ\ $\rightarrow$}
\def\hien{$\swarrow$\ タメ\ $\rightarrow$}
\def\zanretu{$\rightarrow\ \leftarrow\ \rightarrow$}
\def\hado{$\downarrow$ $\searrow$ $\rightarrow$}%%波動コマンド
\def\tatsu{$\downarrow$ $\swarrow$ $\leftarrow$}%%竜巻コマンド
\def\syoryu{$\rightarrow$ $\downarrow$ $\searrow$}%%昇竜コマンド
\def\gyakusyoryu{\leftarrow$ $\downarrow$ $\swarrow$}%%逆昇竜コマンド
\def\yoga{$\leftarrow$ $\swarrow$ $\downarrow$ $\searrow$ $\rightarrow$}%%ヨガフレ
\def\gyakuyoga{$\rightarrow$ $\searrow$ $\downarrow$ $\swarrow$ $\leftarrow$}%%逆ヨガ
\def\tenti{$\rightarrow$ $\searrow$ $\downarrow$ $\swarrow$ $\leftarrow$ $\rightarrow$}%%天地
\def\ryuko{$\downarrow$ $\searrow$ $\rightarrow$ $\searrow$ $\downarrow$ $\swarrow$ $\leftarrow$}%%龍虎
\def\orochi{$\downarrow$ $\swarrow$ $\leftarrow$ $\swarrow$ $\downarrow$ $\searrow$ $\rightarrow$}%%大蛇薙
\def\ue{$\uparrow$}
\def\migiue{\$nearrow$}
\def\hidariue{$\nwarrow$}
\def\sita{$\downarrow$}
\def\hidarisita{$\swarrow$}
\def\migisita{$\searrow$}
\def\migi{$\longrightarrow$}
\def\hidari{$\leftarrow$}
\def\Cancel{$\Longrightarrow$}
\def\DC{DC$\Rightarrow$}
\def\SC{SC$\Rightarrow$}
\def\H\C{HC$\Rightarrow$}
\def\MAX\C{MAX$\Rightarrow$}
\def\starone{★☆☆}
\def\startwo{★★☆}
\def\starthree{★★★}
\def\command#1{$\lbrack$\ #1\ $\rbrack$}
\newcommand{\bhline}[1]{\noalign{\hrule height #1}} 
\newcommand{\bvline}[1]{\vrule width #1}  
%%###
%%タイトルとかの設定
\title{KOF13コンボ一覧}
\author{}
\begin{document}
\bfseries
\mathversion{bold}
\maketitle
\thispagestyle{empty}
\tableofcontents % 目次が出現
\newpage
\part{概要}
\section{このテキストについて}
このテキストは、KOF13CLIMAX/家庭用/Steam
Editionの発動を使わない基礎コンボのほか、覚えておくと得するかもしれない程度の情報が記載されています。

\section{表記について}
ボタンと表記の対応は以下のとおりです。

\begin{screen}

 弱パンチ$\rightarrow$ \A\ \ \ 強パンチ$\rightarrow$ \C
 
 弱キック$\rightarrow$ \B\ \ \ 強キック$\rightarrow$ \D
\end{screen}
\vspace{11pt}b

コンボの繋ぎについては以下のとおりに表記します。
冗長さを回避するため、技名は表記せず、コマンドのみを表記します。
コマンドは、各チームの最初に記載。
\begin{screen}
\begin{tabular}{ll}
 キャンセル無し&\ \ \lbrack\ \ コマンド\ \ \rbrack\ \ $\longrightarrow$\ \ \lbrack\ \ コマンド\ \ \rbrack\ \ \\
 キャンセルする&\ \ \lbrack\ \ コマンド\ \ \rbrack\ \ $\Longrightarrow$\ \ \lbrack\ \ コマンド\ \ \rbrack\ \ \\
 ドライブキャンセル&\ \ \lbrack\ \ コマンド\ \ \rbrack\ \ DC$\Rightarrow$\ \ \lbrack\ \ コマンド\ \ \rbrack\ \ \\
 スーパーキャンセル&\ \ \lbrack\ \ コマンド\ \ \rbrack\ \ SC$\Rightarrow$\ \ \lbrack\ \ コマンド\ \ \rbrack\ \ \\
 $n$段目キャンセル&\ \ \lbrack\ \ コマンド$(n)$\ \ \rbrack\ \ $\Rightarrow$\ \ \lbrack\ \
 コマンド\ \ \rbrack\ \
\end{tabular}
\end{screen}
\begin{itembox}[l]{注意}
\begin{itemize}
\item コマンドは右向き時のものです。
\item 混乱を避けるため位置が途中で入れ替わることコンボであっても、コマンドは右向き時のものを表記しています。
\item コマンドに$(\downarrow$ $\swarrow)$ $\leftarrow$ $\swarrow$ $\downarrow$
$\searrow$ $\rightarrow$ のように()がある場合、その部分は省略可能です。
\end{itemize}
\end{itembox}


\newpage
\part{日本チーム}
\section{コマンド}
\subsection{表京}
%%特殊技
\begin{itembox}[l]{特殊技}
\begin{tabular}{lll}
外式・轟斧\ 陽&$\rightarrow$\ +\ \B&中段\\
八拾八式&$\searrow$\ +\ \D&下段/二段技/初段キャンセル可\\
外式・奈落落とし&(空中で)$\swarrow$\ or\ $\downarrow$\ or\ $\searrow$\ +\ \C&バクステ中も可/中段
\end{tabular}
\end{itembox}
%%必殺技
\begin{itembox}[l]{必殺技}
\begin{tabular}{lll}
百式・鬼焼き&$\rightarrow$ $\downarrow$ $\searrow$\ +\ \A\ or\ \C&弱上半身無敵/強全身無敵\\
百八式・闇払い&$\downarrow$ $\searrow$ $\rightarrow$\ +\ \A\ or\ \C&飛び道具判定/EX版弾貫通\\ 
百壱式・朧車&$\leftarrow$ $\downarrow$ $\swarrow$\ +\ \B\ or\ \D&EX版全身無敵\\
弐百拾弐式・琴月\ 陽&$\rightarrow$ $\searrow$ $\downarrow$ $\swarrow$ $\leftarrow$\ +\ \B\ or\ \D&EX版コマ投げ\\
七拾五式・改&$\downarrow$ $\searrow$ $\rightarrow$\ +\ \B\ or\ \D&二段技/連続ガードではない
\end{tabular}
\end{itembox}
%%超必殺技
\begin{itembox}[l]{超必殺技}
\begin{tabular}{lll}
裏百八式・大蛇薙&$\downarrow$ $\swarrow$ $\leftarrow$ $\swarrow$ $\downarrow$ $\searrow$ $\rightarrow$\ +\ \A\ or\ \C& 空中可/弾貫通/溜め可
\end{tabular}
\end{itembox}
%%NEOMAX超必殺技
\begin{itembox}[l]{NEOMAX超必殺技}
\begin{tabular}{lll}
裏弐百拾壱式・天叢雲&$\downarrow$ $\searrow$ $\rightarrow$ $\downarrow$ $\searrow$ $\rightarrow$\ +\ \A\C&弾貫通/ヒット時のみ演出
\end{tabular}
\end{itembox}
\newpage
\subsection{紅丸}
%%特殊技
\begin{itembox}[l]{特殊技}
\begin{tabular}{lll}
ジャックナイフキック&$\rightarrow$ \ +\ \B&下段無敵\\
フライングドリル&(空中で)$\downarrow$\ +\ \D&バクステからは不可
\end{tabular}
\end{itembox}
%%必殺技
\begin{itembox}[l]{必殺技}
\begin{tabular}{lll}
居合い蹴り&$\downarrow$ $\searrow$ $\rightarrow$ \ +\ \B\ or\ \D&GC\A\B
から確定反撃有り\\
反動三段蹴り&(居合い蹴り中)$\downarrow$ $\uparrow$\ +\ \B\ or\ \D&ディレイ可\\
スーパー稲妻キック&$\rightarrow$ $\downarrow$ $\searrow$\ +\ \B\ or\ \D&弱上半身無敵?/強完全無敵\\
紅丸ランサー&$\downarrow$ $\swarrow$ $\leftarrow$\ +\ \A\ or\ \C&EX版相手をサーチ\\
雷靭拳&$\downarrow$ $\searrow$ $\rightarrow$ \ +\ \A\ or\ \C&空中可\\
紅丸コレダー&$\rightarrow$ $\searrow$ $\downarrow$ $\swarrow$ $\leftarrow$ $\rightarrow$\ +\ \A\ or\ \C&1F投げ
\end{tabular}
\end{itembox}
%%超必殺技
\begin{itembox}[l]{超必殺技}
\begin{tabular}{lll}
雷光拳&$\downarrow$ $\searrow$ $\rightarrow$ $\downarrow$ $\searrow$ $\rightarrow$\ +\ \A\ or\ \C&EX全身無敵\\
紅丸ローリングサンダー&$\downarrow$ $\swarrow$ $\leftarrow$ $\downarrow$ $\swarrow$ $\leftarrow$\ +\ \A\ or\ \C&EX無し/全身無敵
\end{tabular}
\end{itembox}
%%NEOMAX超必殺技
\begin{itembox}[l]{NEOMAX超必殺技}
\begin{tabular}{lll}
雷皇拳&$\downarrow$ $\searrow$ $\rightarrow$ $\downarrow$ $\searrow$ $\rightarrow$\ +\ $\B\ \D$&方向キーで多少軌道が変化
\end{tabular}
\end{itembox}
\newpage
\subsection{大門}
%%特殊技
\begin{itembox}[l]{特殊技}
\begin{tabular}{lll}
頭上払い&$\searrow$\ +\ \C&
\end{tabular}
\end{itembox}
%%必殺技
\begin{itembox}[l]{必殺技}
\begin{tabular}{lll}
天地返し&$\rightarrow$ $\searrow$ $\downarrow$ $\swarrow$ $\leftarrow$ $\rightarrow$\ +\ \A\ or\ \C&1F投げ\\
雲掴み投げ&$\leftarrow$ $\swarrow$ $\downarrow$ $\searrow$ $\rightarrow$\ +\ \A&EX版無敵有り\\
切り株返し&$\leftarrow$ $\swarrow$ $\downarrow$ $\searrow$ $\rightarrow$\ +\ \C&EX版無し/ダウン追い打ち\\
地雷震&$\rightarrow$ $\downarrow$ $\searrow$\ +\ \A&地震判定/EX版初段中段\\
地雷震(フェイント)&$\rightarrow$ $\downarrow$ $\searrow$\ +\ \C&EX版無し/攻撃判定無し\\
超受け身&$\downarrow$ $\swarrow$ $\leftarrow$\ +\ \B\ or\ \D&移動技/強/EX打撃無敵\\
超大外刈り&$\rightarrow$ $\downarrow$ $\searrow$\ +\ \B\ or\ \D&投げ技/発生まで無敵有り
\end{tabular}
\end{itembox}
%%超必殺技
\begin{itembox}[l]{超必殺技}
\begin{tabular}{lll}
地獄極楽落とし&$\rightarrow$ $\searrow$ $\downarrow$ $\swarrow$ $\leftarrow$ $\rightarrow$ $\searrow$ $\downarrow$ $\swarrow$ $\leftarrow$\ +\ \A\ or\ \C&1F投げ
\end{tabular}
\end{itembox}
%%NEOMAX超必殺技
\begin{itembox}[l]{NEOMAX超必殺技}
\begin{tabular}{lll}
驚天動地&$\downarrow$ $\searrow$ $\rightarrow$ $\downarrow$ $\searrow$ $\rightarrow$\ +\ \A\C&当身技/MAXキャンセル時打撃技
\end{tabular}
\end{itembox}
\newpage
\section{コンボ}
\subsection{表京}
\subsubsection{中央コンボ}
\begingroup
 \renewcommand{\arraystretch}{1.2}
\begin{tabular*}{15.1cm}{@{\extracolsep{\fill}}|p{3em}||p{12.9cm}|}\hline
\multicolumn{2}{|p{14.6cm}|}{
\PG{0}\ \ \ \DG{0}
}\\\bhline{2pt}
コンボ&
$\lbrack\ $近\ $\C\ \rbrack$\ (\Cancel\ $\lbrack\ \searrow\ $\D\ $(\
1\ )\rbrack$)
\Cancel\ \command{\hado\ +\ \D}\\
&\migi\ \command{\gyakuyoga\ +\ \B\ or\ \D
}\\\hline
補足&最初に覚えるべき基礎コンボ。ダメージ重視\\\bhline{2pt}
コンボ&
\command{\downarrow \B}\ $\times\ 1\ \sim\ 2$ \Cancel \command{立\B}
\Cancel\ \command{\hado\ +\ \B}\\\hline
補足&最初に覚えるべき小技始動コンボ1。ガード後\ $-1F$\\\bhline{2pt}
コンボ&
\command{\downarrow \B}\ $\times\ 1\ \sim\ 2$ \Cancel \command{立\B}
\Cancel\ \command{\gyakuyoga\ +\ \B\ or\ \D}\\\hline
補足&最初に覚えるべき小技始動コンボ2。ダメージ重視\\\hline\hline
\multicolumn{2}{|p{14.6cm}|}{
\PG{1}\ \ \ \DG{0}
}\\\bhline{2pt}
コンボ&
$\lbrack\ $近\ $\C\ \rbrack$\ (\Cancel\ $\lbrack\ \searrow\ \D\ (\
1\ )\rbrack$)
\Cancel\ \command{\hado\ +\ \D}\\
&\migi\ \command{\gyakusyoryu\ +\ \B\ }\ \migi \ \command{\gyakusyoryu\ +\
\B\D} \\\hline
補足&最初に覚えるべき1ゲージコンボ1。端から端まで運べる\\\hline\hline
\multicolumn{2}{|p{14.6cm}|}{
\PG{2}\ \ \ \DG{0}
}\\\bhline{2pt}
コンボ&
$\lbrack\ $近\ $\C\ \rbrack$\ (\Cancel\ $\lbrack\ \searrow\ \D\ (\
1\ )\rbrack$)
\Cancel\ \command{\hado\ +\ \D}\\
& \migi\ \command{\gyakusyoryu\ +\ \B\ }\ \migi \ \command{\orochi\ +\ \A\C}
\\\hline
補足&倒しきれる時用。タメ不要。端ではダメージ減\\\bhline{2pt}
コンボ&
\command{\downarrow \B}\ $\times\ 1\ \sim\ 2$ \Cancel \command{立\B}
\Cancel\ \command{\hado\ +\ \B}\\
&\migi\ \command{\orochi\ +\ \A\C}\\\hline
補足&倒しきれる時用の小技始動コンボ。タメ不要。端ではダメージ減\\\bhline{2pt}
\end{tabular*}
\endgroup
\newpage
\subsubsection{端コンボ}
\begingroup
 \renewcommand{\arraystretch}{1.2}
\begin{tabular*}{15.1cm}{@{\extracolsep{\fill}}|p{3em}||p{12.9cm}|}\hline
\multicolumn{2}{|p{14.6cm}|}{
\PG{0}\ \ \ \DG{0}
}\\\bhline{2pt}
コンボ&
$\lbrack\ $近\ $\C\ \rbrack$\ (\Cancel\ $\lbrack\ \searrow\ \D\ (\
1\ )\rbrack$)
\Cancel\ \command{\hado\ +\ \D}\\
&\migi\ \command{\gyakusyoryu\ +\ \B}\ \migi\ \command{\hado\ +\ \B}\ \migi\
\command{\gyakusyoryu\ +\ \B}\\
& \migi\ \command{\syoryu\ +\ \A}\\\hline
補足&端の基礎コンボ。できなくてもいいが、ゲージ回収多め\\\bhline{2pt}
コンボ&
\command{\downarrow\ \B}\ $\times\ 1\ \sim\ 2$ \Cancel \command{立\B}
\Cancel\ \command{\hado\ +\ \B}\ \migi\ \command{\syoryu\ +\ \C}\\\hline
補足&端の小技始動基礎コンボ\\\hline\hline
\multicolumn{2}{|p{14.6cm}|}{
\PG{0}\ \ \ \DG{50}
}\\\bhline{2pt}
コンボ&
\command{\downarrow\ \B}\ $\times\ 1\ \sim\ 2$ \Cancel \command{立\B}
\Cancel\ \command{\hado\ +\ \B}\ \migi\ \command{\syoryu\ +\ \C}\\
&\DC\ \command{\hado\ +\ \B}\ \migi\ \command{\gyakusyoryu\ \B}\ \migi\
\command{\syoryu\ +\ \C} \\\hline 
補足&端の小技始動でドライブゲージを使うコンボ\\\hline\hline
\multicolumn{2}{|p{14.6cm}|}{
\PG{1}\ \ \ \DG{50}
}\\\bhline{2pt}
コンボ&
$\lbrack\ $近\ $\C\ \rbrack$\ (\Cancel\ $\lbrack\ \searrow\ \D\ (\
1\ )\rbrack$)
\Cancel\ \command{\hado\ +\ \D}\\
&\migi\ \command{\gyakusyoryu\ +\ \B}\ \SC\ \command{\orochi\ +\A\ or\
\C}\\
&\migi\ \command{\hado\ +\ \A}\ \migi\ \command{\gyakuyoga\ +\B\ or\
\D}\\\hline 
補足&大蛇薙は\command{\yoga\ +\ \A\ or\ \C}で出る\\\hline\hline
\multicolumn{2}{|p{14.6cm}|}{
\PG{2}\ \ \ \DG{50}
}\\\bhline{2pt}
コンボ&
$\lbrack\ $近\ $\C\ \rbrack$\ (\Cancel\ $\lbrack\ \searrow\ \D\ (\
1\ )\rbrack$)
\Cancel\ \command{\hado\ +\ \D}\\
&\migi\ \command{\gyakusyoryu\ +\ \B}\ \migi\ \command{\hado\ +\ \B}\ \migi\
\command{\gyakusyoryu\ +\ \B}\\
&\migi\ \command{\syoryu\ +\ \A}\ \SC\ \command{\orochi\ +\ \A\C}\\
&\migi\ \command{\hado\ +\ \A}\ \migi\ \command{\gyakuyoga\ +\ \B\ or\
\D}\\\hline 補足&弱鬼焼きからの大蛇薙がやや忙しい。\\\bhline{2pt}
\end{tabular*}
\endgroup
\newpage
\newpage
\subsection{紅丸}
\subsubsection{中央コンボ}
\begingroup
 \renewcommand{\arraystretch}{1.2}
\begin{tabular*}{15.1cm}{@{\extracolsep{\fill}}|p{3em}||p{12.9cm}|}\hline
\multicolumn{2}{|p{14.6cm}|}{
\PG{0}\ \ \ \DG{0}
}\\\bhline{2pt}
コンボ&
\command{近 \C\ or\ \D}\ \Cancel\ \command{\tenti\ +\
\A\ or\ \C}\\\hline
補足&最初に覚えるべき基礎コンボ。入れ込み安全。\\\bhline{2pt}
コンボ&
\command{\downarrow\ \B}\ $\times\ 1\ \sim\ 2$\ \Cancel\ \command{\hado\ +\
\B\ or\ \D}\ \Cancel\ \command{\downarrow\ \uparrow\ +\ \B\ or\
\D}\\\hline
補足&最初に覚えるべき小技始動基礎コンボ。居合い蹴りまで出してノーゲージの反撃は少ない。\command{近\ \C}、\command{近\
\D}で始動してもいい\\\hline\hline
\multicolumn{2}{|p{14.6cm}|}{
\PG{1}\ \ \ \DG{0}
}\\\bhline{2pt}
コンボ&
\command{近 \C\ or\ \D}\ \Cancel\ \command{\tenti\ +\A\C}\ \migi
\command{\tatsu\ \A\ or\ \C}\\
&\ \migi\ \command{\hado\ +\ \B\ or\ \D}\ \Cancel\ \command{\downarrow\
\uparrow\ +\ \B\ or\ \D}
\\\hline
補足&EX紅丸コレダー後は紅丸ランサーで追撃可。居合い蹴りを遅らせるとダメージが増える\\\bhline{2pt}
コンボ&
\command{\downarrow\ \B}\ $\times\ 1\ \sim\ 2$\ \Cancel\ \command{\hado\ +\
\A\C}\ \migi\ \command{\hado\ +\
\B\ or\ \D}\\
& \Cancel\ \command{\downarrow\ \uparrow\ +\ \B\ or\
\D\ }
\\\hline
補足&EX雷靭拳後、前転か前中Jなどを挟むと運ぶ距離が増える\\\hline\hline
\multicolumn{2}{|p{14.6cm}|}{
\PG{0}\ \ \ \DG{50}
}\\\bhline{2pt}
コンボ&
\command{\downarrow\ \B}\ $\times\ 1\ \sim\ 2$\ \Cancel\ \command{\hado\ +\
\B\ or\ \D}\ \Cancel\ \command{\downarrow\ \uparrow\ +\ \B\ or\
\D\ (\ 1\ )\ }\\
&\DC\ \command{\tatsu\ \A\ or\ \C}\ \migi\ \command{\hado\ +\
\B\ or\ \D}\ \Cancel\ \command{\downarrow\ \uparrow\ +\ \B\ or\
\D\ }\\\hline
補足&ドライブゲージを使うコンボの場合は居合い蹴り~反動三段蹴りを使う。紅丸ランサー後の居合い蹴りを遅らせるとダメージが増える。
\command{近\
\C}、\command{近\ \D}で始動してもいい\\\hline\hline
\multicolumn{2}{|p{14.6cm}|}{
\PG{1}\ \ \ \DG{50}
}\\\bhline{2pt}
コンボ&
\command{\downarrow\ \B}\ $\times\ 1\ \sim\ 2$\ \Cancel\ \command{\hado\ +\
\B\ or\ \D}\ \Cancel\ \command{\downarrow\ \uparrow\ +\ \B\ or\
\D\ (\ 1\ )\ }\\
&\DC\ \command{\tenti\ \A\C}\ \migi\ \command{\tatsu\ +\ \A\ or\ \C}\\
&\migi\ \command{\hado\ +\
\B\ or\ \D}\ \Cancel\ \command{\downarrow\ \uparrow\ +\ \B\ or\
\D\ }\\\hline
補足&上記2つのコンボの複合系。
\command{近\
\C}、\command{近\ \D}で始動してもいい\\\hline\hline
\multicolumn{2}{|p{14.6cm}|}{
\PG{2}\ \ \ \DG{50}
}\\\bhline{2pt}
コンボ&
\command{\downarrow\ \B}\ $\times\ 1\ \sim\ 2$\ \Cancel\ \command{\hado\ +\
\B\ or\ \D}\ \Cancel\ \command{\downarrow\ \uparrow\ +\ \B\ or\
\D\ (\ 1\ )\ }\\
&\DC\ \command{\tatsu\ \A\ or\ \C}\ \migi\ \command{\hado\ \hado\ +\
\A\C}\\\hline
補足&\command{近\
\C}、\command{近\ \D}で始動してもいい\\\bhline{2pt}
\end{tabular*}
\endgroup
%%一つのクラス
\newpage
\subsubsection{端コンボ}
\begingroup
 \renewcommand{\arraystretch}{1.2}
\begin{tabular*}{15.1cm}{@{\extracolsep{\fill}}|p{3em}||p{12.9cm}|}\hline
\multicolumn{2}{|p{14.6cm}|}{
\PG{0}\ \ \ \DG{0}
}\\\bhline{2pt}
コンボ&
\command{(端に向かって)通常投げ}\ \migi\ \command{\hado\ +\
\B\ or\ \D}\ \Cancel\ \command{\downarrow\ \uparrow\ +\ \B\ or\
\D\ }\\\hline
補足&端に向かって投げた場合は追撃が可能\\\hline\hline
\multicolumn{2}{|p{14.6cm}|}{
\PG{0}\ \ \ \DG{50}
}\\\bhline{2pt}
コンボ&
\command{(端に向かって)通常投げ}\ \migi\ \command{\hado\ +\
\B\ or\ \D}\ \Cancel\ \command{\downarrow\ \uparrow\ +\ \B\ or\
\D\ (\ 1\ )}\\
&\DC\ \command{\tatsu\ +\ \A\ or\ \C}\ \migi\ \command{\hado\ +\ \A\ or\
\C}\  \migi\ \command{\hado\ +\ \B\ or\
\D}\ \migi\ \command{\downarrow\ \uparrow\ +\ \B\ or\ \D\ }\\\hline
補足&特になし
\\\hline\hline
\multicolumn{2}{|p{14.6cm}|}{
\PG{1}\ \ \ \DG{0}
}\\\bhline{2pt}
コンボ&
\command{(端に向かって)通常投げ}\ \migi\ \command{\tatsu\ \tatsu\ +\ \A\ or\
\C}\\\hline\hline
\multicolumn{2}{|p{14.6cm}|}{
\PG{1}\ \ \ \DG{50}
}\\\bhline{2pt}
コンボ&
\command{(端に向かって)通常投げ}\ \migi\ \command{\hado\ +\
\B\ or\ \D}\\
&\Cancel\ \command{\downarrow\ \uparrow\ +\ \B\ or\
\D\ (\ 1\ )}\\
&\DC\ \command{\hado\ +\A\C}\ \migi\ \command{\tatsu\ +\ \A\ }\ \migi\
\command{\tatsu\ +\ \A\ }\\
&\migi\ \command{\hado\ +\ \C\ }\ \migi\ \command{\syoryu\ +\ \B\ }\\\hline
補足&特になし\\
\hline\hline
\multicolumn{2}{|p{14.6cm}|}{
\PG{2}\ \ \ \DG{0}
}\\\bhline{2pt}
コンボ&
\command{(端に向かって)通常投げ}\ \migi\ \command{\hado\ \hado\ +\ \A\C}\\\hline\hline
\multicolumn{2}{|p{14.6cm}|}{
\PG{2}\ \ \ \DG{50}
}\\\bhline{2pt}
コンボ&
\command{(端に向かって)通常投げ}\ \migi\ \command{\hado\ +\
\B\ or\ \D}\\
& \Cancel\ \command{\downarrow\ \uparrow\ +\ \B\ or\
\D\ (\ 1\ )}\\
&\DC\ \command{\hado\ +\A\C}\ \migi\ \command{\tatsu\ +\ \A\ }\ \migi\
\command{\tatsu\ +\ \A\ }\\
&\migi\ \command{\hado\ +\ \C\ }\ \migi\ \command{\syoryu\ +\ \B\D\
}\\\hline
補足&特になし\\\hline
\end{tabular*}
\endgroup
\newpage
\subsection{大門}
\subsubsection{中央コンボ}
\begingroup
 \renewcommand{\arraystretch}{1.2}
\begin{tabular*}{15.1cm}{@{\extracolsep{\fill}}|p{3em}||p{12.9cm}|}\hline
\multicolumn{2}{|p{14.6cm}|}{
\PG{0}\ \ \ \DG{0}
}\\\bhline{2pt}
コンボ&
\command{近\ \C}\ (\ \Cancel\ \command{$\searrow$\ \C}\ )\ (\ \Cancel\
\command{\tatsu\ +\ \B}\ )\\
&\Cancel\ \command{\tenti\ +\ \A\ or\ \C}\\\hline
補足&基礎コンボ。ダメージが高い。遠いと\command{$\searrow$\
\C}が当たらない。弱超受け身は動作中必殺技以上でキャンセルできるので、ゲージをためられる\\\bhline{2pt} コンボ&
\command{立\ \A}\ $\times\ 1\ \sim\ 2$ \Cancel\ \command{立\ \B}\ (\ \Cancel\
\command{\tatsu\ +\ \B}\ )\\
&\Cancel\ \command{\tenti\ +\ \A\ or\ \C}\\\hline
補足&小技始動基本コンボ。弱超受け身は動作中必殺技以上でキャンセルできるので、ゲージをためられる\\\bhline{2pt}
コンボ&
\command{$J$\C\D}(カウンターヒット)\ \migi \ \command{$\searrow$\ \C}\ \Cancel
\command{\yoga\ +\ \A}
\\\hline
補足&ジャンプふっ飛ばしがカウンターヒットすると壁バウンドを誘発するので落ち着いて追撃\\\hline\hline
 \multicolumn{2}{|p{14.6cm}|}{
\PG{1}\ \ \ \DG{0}
}\\\bhline{2pt}
コンボ&
\command{近\ \C}\ (\ \Cancel\ \command{$\searrow$\ \C}\ )\
(\ \Cancel\ \command{\tatsu\ +\ \B}\ )\ \\
&\Cancel\ \command{\gyakuyoga\ $\times\ 2$\
+\ \A\ or\ \C}\\\hline
補足&弱超受け身は動作中必殺技以上でキャンセルできるので、コマンドを分割できる。できなくてもいい\\\bhline{2pt}
コンボ&
\command{立\ \A}\ $\times\ 1\ \sim\ 2$ \Cancel\ \command{立\ \B}\ (\ \Cancel\ \command{\tatsu\ +\ \B}\ )\ \\
&\Cancel\ \command{\gyakuyoga\ $\times\ 2$\
+\ \A\ or\ \C}\\\hline
補足&小技始動基本コンボ。弱受け身については同上。できなくてもいい\\\bhline{2pt}
\multicolumn{2}{|p{14.6cm}|}{
\PG{1}\ \ \ \DG{50}
}\\\bhline{2pt}
コンボ&\command{\syoryu\ +\ \B\ or\ \D}\ \SC\ \command{\gyakuyoga\ $\times\ 2$\
+\ \A\ or\ \C}\\\hline
補足&投げ始動。超大外刈はスーパーキャンセルで地獄極楽落としに繋げられる\\\bhline{2pt}
コンボ&
\command{$J$\C\D}(カウンターヒット)\ \migi \ \command{$\searrow$\ \C}\ \Cancel
\command{\yoga\ +\ \A}\\
&\SC\ \command{\gyakuyoga\ $\times\ 2$\ +\ \A\ or\ \C}
\\\hline
補足&雲掴み投げからも地獄極楽落としに繋げられる\\\hline\hline
\multicolumn{2}{|p{14.6cm}|}{
\PG{2}\ \ \ \DG{50}
}\\\bhline{2pt}
コンボ&\command{\syoryu\ +\ \B\ or\ \D}\ \SC\ \command{\gyakuyoga\ $\times\ 2$\
+\ \A\C}\\\hline
補足&投げ始動。2ゲージ使う場合はEX地獄極楽落としに\\\bhline{2pt}
コンボ&
\command{$J$\C\D}(カウンターヒット)\ \migi \ \command{$\searrow$\ \C}\ \Cancel
\command{\yoga\ +\ \A}\\
&\SC\ \command{\gyakuyoga\ $\times\ 2$\ +\ \A\C}
\\\hline
補足&同上\\\bhline{2pt}
\end{tabular*}
\endgroup
\newpage
\subsubsection{端コンボ}
\begingroup
 \renewcommand{\arraystretch}{1.2}
\begin{tabular*}{15.1cm}{@{\extracolsep{\fill}}|p{3em}||p{12.9cm}|}\hline
\multicolumn{2}{|p{14.6cm}|}{
\PG{0}\ \ \ \DG{0}
}\\\bhline{2pt}
コンボ&
\command{$J$\C\D}\ \migi\ \command{\yoga\ +\ \C}\\\hline
補足&ジャンプふっ飛ばしが通常ヒットした場合スライドダウンを誘発する。
切り株返しはダウン追撃判定があるのでコンボになる。ただし位置が入れ替わる\\\hline\hline
\multicolumn{2}{|p{14.6cm}|}{
\PG{0}\ \ \ \DG{50}
}\\\bhline{2pt}
コンボ&
\command{$J$\C\D}\ \migi\ \command{\yoga\ +\ \C}\ \DC\ \command{\tatsu\
+\ \B}\\& \Cancel\ \command{\yoga\ +\ \C$}\\\hline
補足&弱超受け身で移動し、動作を切り株返しでキャンセルする。最初の切り株返しで位置が入れ替わるので注意。ダメージは小さいが画面端に戻せる。
\\\hline\hline
\multicolumn{2}{|p{14.6cm}|}{
\PG{1}\ \ \ \DG{50}
}\\\bhline{2pt}
コンボ&
\command{$J$\C\D}\ \migi\ \command{\yoga\ +\ \C}\ \\
&\SC\
\command{\gyakuyoga\ $\times\ 2$\ +\ \A\ or\ \C}\\\hline
補足&位置は入れ替わるが地獄極楽落としで追撃が可能 \\\hline\hline
\multicolumn{2}{|p{14.6cm}|}{
\PG{2}\ \ \ \DG{50}
}\\\bhline{2pt}
コンボ&
\command{$J$\C\D}\ \migi\ \command{\yoga\ +\ \C}\\
&\SC\ \command{\gyakuyoga\ $\times\ 2$\ +\ \A\C}\\\hline
補足&同上
\\\bhline{2pt}
\end{tabular*}
\endgroup
\newpage
\part{餓狼チーム}
\section{コマンド}
\subsection{テリー}
\begin{itembox}[l]{特殊技}
\begin{tabular}{lll}
ライジングアッパー&$\searrow$\ +\ \C&\\
バックナックル&$\rightarrow$\ +\ \A&\\
ターゲットコンボ?&$\downarrow$ \A \Cancel $\downarrow$ \C
\end{tabular}
\end{itembox}
\begin{itembox}[l]{必殺技}
\begin{tabular}{lll}
パワーウェイブ&\hado\ +\ \A\ or\ \C&弾判定/EX版弾貫通\\
バーンナックル&\tatsu\ +\ \A\ or\ \C&\\
クラックシュート&\tatsu \B\ or\ \D&EX版中段\\
ライジングタックル&$\downarrow$\ タメ\ $\uparrow$\ +\ \A\ or\ \C&弱上半身無敵/強完全無敵
\end{tabular}
\end{itembox}
\begin{itembox}[l]{超必殺技}
\begin{tabular}{lll}
パワーゲイザー&\tatsu $\swarrow$ $\rightarrow$\ +\ \A\ or\ \C&EX版カスヒット時追撃可能\\
バスターウルフ&\hado\hado\ +\ \B\ or\ \D&カスヒット時追撃可能
\end{tabular}
\end{itembox}
\begin{itembox}[l]{NEOMAX超必殺技}
\begin{tabular}{lll}
トリニティゲイザー&\hado\hado\ +$\A\C$&ヒット数不安定(2~4Hit)\\
&&3Hit以上した場合追撃可
\end{tabular}
\end{itembox}
\newpage
\subsection{アンディ}
\begin{itembox}[l]{特殊技}
\begin{tabular}{lll}
平手打ち&$\rightarrow$\ +\ \A&\\
ターゲットコンボ?&立\B\ \Cancel\ 立\D&必殺技キャンセル可
\end{tabular}
\end{itembox}
\begin{itembox}[l]{必殺技}
\begin{tabular}{lll}
斬影拳&$\swarrow$ $\leftarrow$\ +\ \A\ or\ \C&相手の位置に応じて発生が変化\\
空破弾&\yoga\ +\ \B\ or\ \D&EX版ガクラ値高い\\
空破弾ブレーキング&(通常空破弾中)$\B \D$&\\
飛翔拳&\tatsu\ +\ \A\ or\ \C&EX版弾貫通\\
昇龍弾&\syoryu\ +\ \A\ or\ \C&弱上半身無敵/強完全無敵
\end{tabular}
\end{itembox}
\begin{itembox}[l]{超必殺技}
\begin{tabular}{lll}
超裂破弾&\orochi\ +\ \B\ or\ \D&高めヒット時追撃可\\
絶・飛翔拳&\ryuko\ +\ \A\ or\ \C&EX版無し/最終段スライドダウン
\end{tabular}
\end{itembox}

\begin{itembox}[l]{NEOMAX必殺技}
\begin{tabular}{lll}
超☆神☆速☆斬影拳&\orochi\ +\ \A\C&発生と同時に無敵が切れる?\\
&&空中可
\end{tabular}
\end{itembox}
\newpage
\subsection{ジョー}
\begin{itembox}[l]{特殊技}
\begin{tabular}{lll}
ステップインミドルキック&$\rightarrow$\ +\ \B&\\
スライディング&$\searrow$\ +\ \B&下段/爆裂拳でのみキャンセル可
\end{tabular}
\end{itembox}
\begin{itembox}[l]{必殺技}
\begin{tabular}{lll}
ハリケーンアッパー&\yoga\ +\ \A\ or\ \C&EX版弾貫通 \times 3\\
タイガーキック&\syoryu\ +\ \B\ or\ \D&EX版デカキャラ以外の\\
&&しゃがみに二段目当たらず\\
スラッシュキック&\yoga\ +\ \B\ or\ \D&弱先端ガード時有利\\
爆裂拳&\A\ or\ \C\ 連打&最速止めガード時有利\\
爆裂フィニッシュ&(爆裂拳中)\tatsu\ +\ \A\ or\ \C&最終段スタン値10\\
黄金のカカト&\tatsu\ +\ \B\ or\ \D&EX版ガードさせて有利
\end{tabular}
\end{itembox}
\begin{itembox}[l]{超必殺技}
\begin{tabular}{lll}
スクリューアッパー&\hado\hado\ +\ \A\ or\ \C&弾貫通\\
爆裂ハリケーンタイガーカカト&\ryuko\ +\ \A\ or\ \C&EX無し
\end{tabular}
\end{itembox}
\begin{itembox}[l]{NEOMAX超必殺技}
\begin{tabular}{lll}
スクリューストレート&\hado\hado\ +\ $\B\ \D$&最終段以外ガーキャン不可
\end{tabular}
\end{itembox}
\newpage
\section{コンボ}
\subsection{テリー}
\subsubsection{中央コンボ}
\begingroup
 \renewcommand{\arraystretch}{1.2}
\begin{tabular*}{15.1cm}{@{\extracolsep{\fill}}|p{3em}||p{12.9cm}|}\hline
\multicolumn{2}{|p{14.6cm}|}{
\PG{0}\ \ \ \DG{0}
}\\\bhline{2pt}
コンボ&
\command{近\ \D}\ \Cancel\ \command{$\searrow$\ +\ \C}\ \Cancel\
\command{\tatsu\ +\ \A}\\\hline 補足&基本コンボ。始動は\command{近\ \C}でもOK\\\bhline{2pt}
コンボ&
\command{$\downarrow$\ \B}\ \Cancel\command{$\downarrow$\ \A}\ \Cancel\
\command{$\downarrow$\ \C}\ \Cancel\ \command{\tatsu\ +\ \A}\\\hline
補足&小技始動基本コンボ。以下すべてのコンボはこの始動でも可\\\hline\hline
\multicolumn{2}{|p{14.6cm}|}{
\PG{0}\ \ \ \DG{50}
}\\\bhline{2pt}
コンボ&
\command{近\ \D}\ \Cancel\ \command{$\searrow$\ +\ \C}\ \Cancel\
\command{\tatsu\ +\ \A}\ \DC\ \command{\tatsu\ +\ \B}\ \migi\ \command{遠\ \C}
\\\hline
補足&\command{遠\ \C}は連打でOK。それほどダメージは伸びないので倒しきれる時用。
\\\hline\hline
\multicolumn{2}{|p{14.6cm}|}{
\PG{1}\ \ \ \DG{0}
}\\\bhline{2pt}
コンボ&
\command{近\ \D}\ \Cancel\ \command{$\searrow$\ +\ \C}\ \Cancel\
\command{\tatsu\ $\swarrow\ \rightarrow$\ +\ \A\ or\ \C}\\\hline
補足&始動は\command{近\ \C}でもOK。分割がきかないのでやや忙しい\\\hline\hline
\multicolumn{2}{|p{14.6cm}|}{
\PG{1}\ \ \ \DG{50}
}\\\bhline{2pt}
コンボ&
\command{近\ \D}\ \Cancel\ \command{$\searrow$\ +\ \C}\ \Cancel\
\command{\tatsu\ +\ \A}\\
&\DC\ \command{\tatsu\ +\ \B}\ \migi\ \command{\vtame\ +\A\C}
\\\hline
補足&\command{\vtame\ +\A\C}は\command{\hado\ \hado\ +\ \B\ or\
\D}で代用可能。\\\hline\hline
\multicolumn{2}{|p{14.6cm}|}{
\PG{2}\ \ \ \DG{0}
}\\\bhline{2pt}
コンボ&
\command{近\ \D}\ \Cancel\ \command{$\searrow$\ +\ \C}\ \Cancel\
\command{\tatsu\ $\swarrow\ \rightarrow$\ +\ \A\C}\\\hline
補足&始動は\command{近\ \C}でもOK。分割がきかないのでやや忙しい。ダメージは高い\\\bhline{2pt}
\end{tabular*}
\endgroup
\newpage
\subsubsection{端コンボ}
\begingroup
 \renewcommand{\arraystretch}{1.2}
\begin{tabular*}{15.1cm}{@{\extracolsep{\fill}}|p{3em}||p{12.9cm}|}\hline
\multicolumn{2}{|p{14.6cm}|}{
\PG{0}\ \ \ \DG{50}
}\\\bhline{2pt}
コンボ&
\command{近\ \D}\ \Cancel\ \command{$\searrow$\ +\ \C}\ \Cancel\
\command{\tatsu\ +\ \A}\\
&\DC\ \command{\tatsu\ +\ \B}\ \migi\ \command{\vtame\ +\C}
\\\hline
補足&端であれば\command{\vtame\ +\C}で追撃が可能\\\hline\hline
\multicolumn{2}{|p{14.6cm}|}{
\PG{2}\ \ \ \DG{50}
}\\\bhline{2pt}
コンボ&
\command{近\ \D}\ \Cancel\ \command{$\searrow$\ +\ \C}\ \Cancel\ \command{\tatsu\
+\ \A}\\
& \SC\ \command{$\tatsu\ \swarrow\ \rightarrow$\ +\ \A\C}\ \migi\
\command{\vtame\ +\C}
%%コンボルート
\\\hline
補足&
\command{$\tatsu\ \swarrow\ \rightarrow$\ +\ \A\C}は端だとカスアタリだが追撃が可能。
%%補足
\\\bhline{2pt}%%\hline\hline
\end{tabular*}
\endgroup

\newpage
\subsection{アンディ}
\begin{itembox}[l]{基本コンボ始動技}
\begin{tabular}{ll}
$\lbrack$\ 近\
$\C(1)\ \rbrack$\ or\ $\lbrack$\ 近\ $\C\ \rbrack$\ or\ $\lbrack$\ 近\ $\D\
\rbrack$\ \Cancel\ $\lbrack\ \rightarrow\ \A\ \rbrack$&$\rightarrow$\
(\rnum{1})\\
$\lbrack\ \downarrow\ \B\ \rbrack$\ \Cancel\ $(\ \lbrack\ \downarrow\ \A\
\rbrack$\ or\ $\lbrack\ \downarrow\ \B\ \rbrack\ )$\Cancel\ $\lbrack\ \rightarrow\ \A\
\rbrack$&$\rightarrow$\ (\rnum{2})\\
\end{tabular}
\end{itembox}
\newpage
\subsection{ジョー}
\begin{itembox}[l]{基本コンボ始動技}
\begin{tabular}{ll}
$\lbrack$\ 近\ $\C\ \rbrack$\ or\ $\lbrack$\ 立\ $\D\ \rbrack$\ \Cancel\ $\lbrack\
\rightarrow\ \B\ \rbrack$ &$\rightarrow$\ (\rnum{1})\\
$\lbrack$\ 立\ $\A\ \rbrack$\ or\ $\lbrack\ \downarrow\ \B\ \rbrack\ \times\ 1
\sim\ 2$\ \Cancel\ $\lbrack$\ 立\ $\A\ \rbrack$&$\rightarrow$\ (\rnum{2})
\end{tabular}
\end{itembox}
\newpage
\part{サイコソルジャーチーム}
\section{コマンド}
\subsection{アテナ}
\begin{itembox}[l]{特殊技}
\begin{tabular}{lll}
フェニックスボム&$\rightarrow$\ +\ \B&\\
フェニックスボム(空中)&(空中で)$\rightarrow$\ +\ \B&中段/バクステ中も可
\end{tabular}
\end{itembox}
\begin{itembox}[l]{必殺技}
\begin{tabular}{lll}
サイコボールアタック&\hado\ +\ \A\ or\ \C&EX版弾貫通\\
サイコソード&\syoryu\ +\ \A\ or\ \C&強完全無敵\\
サイコリフレクター&\tatsu\ +\ \B\ or\D&弾反射\\
フェニックスアロー&(空中で)\tatsu\ +\ \B\ or\ \D&バクステ中も可\\
サイキックテレポート&\hado\ +\ \B\ or\ \D&EX版途中Cancel可\\
スーパーサイキックスルー&\yoga\ +\ \A\ or\ \C&投げ技/発生まで無敵
\end{tabular}
\end{itembox}
\begin{itembox}[l]{超必殺技}
\begin{tabular}{lll}
シャイニングクリスタルビット&\gyakuyoga\ \times\ 2\ +\ \A\ or\ \C&空中可
\end{tabular}
\end{itembox}
\begin{itembox}[l]{NEOMAX超必殺技}
\begin{tabular}{lll}
サイコメドレー13&\tenti\ +\ \A\C&
\end{tabular}
\end{itembox}
\newpage
\subsection{ケンスウ}\begin{itembox}[l]{特殊技}
\begin{tabular}{lll}
バク転&$\B\ \D$&前半完全無敵/途中キャンセル可\\
ターゲットコンボ?&立\B\ \Cancel\ \C&
\end{tabular}
\end{itembox}
\begin{itembox}[l]{必殺技}
\begin{tabular}{lll}
龍倒打&\hado\ +\ \A\ or\ \C&鱗徹掌へ派生\\
鱗徹掌&(龍倒打中)\hado\ +\ \A\ or\ \C&疾走斬龍脚へ派生\\
疾走斬龍脚&\hado\ +\ \B\ or\ \D&単発のみEX版有り\\&&
鱗徹掌からも派生可\\
龍撲鼓&\syoryu\ +\ \A\ or\ \C&連打可/最終段で疲れる\\
龍顎砕&\gyakusyoryu\ +\ \B\ or\ \D&強完全無敵\\
龍爪撃&(空中で)\tatsu\ +\ \A\ or\ \C&ヒット時着地キャンセル可\\
超球弾&\tatsu\ +\ \A\ or\ \C&EX版弾貫通
\end{tabular}
\end{itembox}
\begin{itembox}[l]{超必殺技}
\begin{tabular}{lll}
神龍・超球弾&\tatsu\tatsu\ +\ \A\ or\ \C&弾貫通\\
神龍凄煌拳&\ryuko\ +\ \A\ or\ \C&1F投げ/EX版無し
\end{tabular}
\end{itembox}
\begin{itembox}[l]{NEOMAX超必殺技}
\begin{tabular}{lll}
醒眼・仙氣発勁&\hado\hado\ +\ \A\C&最終段以外ガーキャン不可
\end{tabular}
\end{itembox}
\newpage
\subsection{チン}
\begin{itembox}[l]{特殊技}
\begin{tabular}{lll}
座盤鉄&$\downarrow$\ $\downarrow$\ +\ \A\ or\ \C&構え/月牙双手へ派生可\\
張果老&$\downarrow\ \downarrow$\ +\ \B\ or\ \D&構え/$\B\ \D$で解除可/遠\Dから\Dおしっぱでも可\\
烏龍絞柱&$\searrow$\ +\ \D&中段/通常技キャンセル時も中段/月牙双手へ派生可
\end{tabular}
\end{itembox}
\begin{itembox}[l]{必殺技}
\begin{tabular}{lll}
酔歩&\tatsu\ +\ \B\ or\ \D&当身/弱上半身/強下半身\\
&&成立時キャンセル可\\
月牙跳激&\tatsu\ +\ \A\ or\ \C&強上半身無敵\\
二起脚&$\searrow$\ +\ \B&EX無し\\
&&\B追加入力で二段目へ派生\\
月牙双手&(烏龍絞柱or座盤鉄中)$\downarrow\ \downarrow$\ +\ \A\ or\ \C&キャンセル可\\
飲酒&\gyakusyoryu\ +\ \A\ or\ \C&EX無し/酒+1\\
&&正確に入力しないと出ない\\
回転的空突&\yoga\ +\ \B\ or\ \D&EX版途中無敵
\end{tabular}
\end{itembox}
\begin{itembox}[l]{超必殺技}
\begin{tabular}{lll}
騰空飛天砲&\hado\hado\ +\ \A\ or\ \C\\
鉄山靠&\orochi\ +\ \A\ or\ \C&EX無し\\
&&弱スライドダウン/強壁バウンド\\
\end{tabular}
\end{itembox}
\begin{itembox}[l]{NEOMAX超必殺技}
\begin{tabular}{lll}
酔操・轟欄炎炮&\hado\hado\ +\ $\B\ \D$&方向キーで向き変化/空中可
\end{tabular}
\end{itembox}
\newpage
\section{コンボ}
\subsection{アテナ}
\begin{itembox}[l]{基本コンボ始動技}
\begin{tabular}{ll}
$\lbrack$\ 近\ $\C\ \rbrack$\ or\ $\lbrack$\ 屈\ $\C\ \rbrack$\ \Cancel\ $\lbrack\
\rightarrow\ \B\ \rbrack$ &$\rightarrow$\ (\rnum{1})\\
$\lbrack$\ 立\ $\A\ \rbrack$\ or\ $\lbrack\ \downarrow\ \B\ \rbrack\ \times\ 1
\sim\ 2$&$\rightarrow$\ (\rnum{2})
\end{tabular}
\end{itembox}
\newpage
\subsection{ケンスウ}
\begin{itembox}[l]{基本コンボ始動技}
\begin{tabular}{ll}
$\lbrack$\ 近\ $\C\ \rbrack$\ or\ $\lbrack$\ 近\ $\D\ \rbrack$&$\rightarrow$\
(\rnum{1})\\
$\lbrack\ \downarrow\ \B\ \rbrack\ \times\ 1
\sim\ 3$&$\rightarrow$\ (\rnum{2})
\end{tabular}
\end{itembox}
\newpage
\subsection{チン}
\begin{itembox}[l]{基本コンボ始動技}
\begin{tabular}{ll}
$\lbrack\ $近\ $\C\ \rbrack$\ (\ \Cancel\ $\lbrack\ \downarrow\ \downarrow\ \B\
\D\ \rbrack$\ $\longrightarrow\ \lbrack\ $遠\ $\C\ \rbrack$\ )&\ $\rightarrow$\
(\rnum{1})\\
$\lbrack\ \downarrow\ \B\ \rbrack$\ \Cancel\ $\lbrack\ $立\ or\ $\downarrow\ \A\
\rbrack$\ (\ \Cancel\ $\lbrack\ \downarrow\ \B\ \rbrack$\ \Cancel\ $\lbrack\ $立\ or\ $\downarrow\ \A\
\rbrack$\ )&\ $\rightarrow$\ (\rnum{2})
\end{tabular}
\end{itembox}
\newpage
\part{怒チーム}
\section{コマンド}
\subsection{レオナ}
\begin{itembox}[l]{特殊技}
\begin{tabular}{lll}
ストライクアーチ&$\rightarrow$\ +\ \B&中段/空中判定\\
\end{tabular}
\end{itembox}
\begin{itembox}[l]{必殺技}
\begin{tabular}{lll}
ボルテックランチャー&\htame\ +\ \A\ or\ \C&弱/強最終段のみGC可\\
ムーンスラッシャー&\vtame\ +\ \A\ or\ \C&弱/強上半身無敵\\
イヤリング爆弾&\tatsu\ +\ \B\ or\ \D&EX版弾貫通\\
グランドセイバー&\htame\ +\ \B\ or\ \D&相手の近くで攻撃発生\\
Xキャリバー&(空中で)\tatsu\ +\ \A\ or\ \C&
\end{tabular}
\end{itembox}
\begin{itembox}[l]{超必殺技}
\begin{tabular}{lll}
Vスラッシャー&(空中で)&突進中無敵\\
&\ryuko\ +\ \A\ or\ \C&EX版のみ\\
&&MAXキャンセル可\\
スラッシュセイバー&\orochi\ +\ \B\ or\ \D&EX版無し
\end{tabular}
\end{itembox}
\begin{itembox}[l]{NEOMAX超必殺技}
\begin{tabular}{lll}
レオナブレード&\orochi\ +\ \A\C&先端ヒット時演出なし
\end{tabular}
\end{itembox}
\newpage
\subsection{ラルフ}
\begin{itembox}[l]{特殊技}
\begin{tabular}{lll}
ジェットアッパー&\searrow\ +\ \A&
\end{tabular}
\end{itembox}
\begin{itembox}[l]{必殺技}
\begin{tabular}{lll}
バーニングハンマー&\hado\ +\ \A\ or\ \C&EX版全身ガードポイント\\
爆弾ラルフパンチ&\tatsu\ +\ \A\ or\ \C&低姿勢技\\
バルカンパンチ&\A\ or\ \C 連打&連打中前進のみ可\\
ガトリングアタック&\htame\ +\ \A\ or\ \C&\\
急降下爆弾パンチ&(空中で)\hado\ +\ \A\ or\ \C&パンチ部分中段
\end{tabular}
\end{itembox}
\begin{itembox}[l]{超必殺技}
\begin{tabular}{lll}
ギャラクティカファントム&\hado\hado\ +\ \A\ or\ \C &最大タメガー不\\
馬乗りバルカンパンチ&\orochi\ +\ \B\ or\ \D&EX版無し
\end{tabular}
\end{itembox}
\begin{itembox}[l]{NEOMAX超必殺技}
\begin{tabular}{lll}
JET・バルカンパンチ&\gyakuyoga\gyakuyoga\ +\ \A\C&
\end{tabular}
\end{itembox}
\newpage
\subsection{クラーク}
\begin{itembox}[l]{特殊技}
\begin{tabular}{lll}
ジェットアッパー&$\searrow$\ +\ \A&\\
ステップ&$\rightarrow$\ +\ $\B\ \D$&移動技
\end{tabular}
\end{itembox}
\begin{itembox}[l]{必殺技}
\begin{tabular}{lll}
SAB&\yoga\ +\ \B\ or\ \D&投げ技/弱/EX1F投げ\\
&&強発生まで全身GP\\
&&+投げ無敵\\
フラッシングエルボー&(SAB中)&強度はSABに依存\\
&\hado\ +\ \A\ or\ \C&EX版スーパーキャンセル可\\
バルカンパンチ&\A\ or\ \C 連打&\\
ガトリングアタック&\htame\ +\ \A\ or\ \C&EX版弾無敵+上半身無敵\\
デスレイクドライブ&(強/EXガトリング中)&\\
&\syoryu\ +\ \A\ or\ \C&\\
\end{tabular}
\end{itembox}
※SAB…スーパーアルゼンチンバックブリーカー\\
※DD…デスレイクドライブ\\
\begin{itembox}[l]{超必殺技}
\begin{tabular}{lll}
UAB&\gyakuyoga\gyakuyoga\ +\ \A\ or\ \C&1F投げ
\end{tabular}
\end{itembox}
※UAB…ウルトラアルゼンチンバックブリーカー\\
\begin{itembox}[l]{NEOMAX超必殺技}
\begin{tabular}{lll}
ウルトラクラークバスター&\gyakuyoga\gyakuyoga\ +\ $\B\ \D$&
\end{tabular}
\end{itembox}
\newpage
\section{コンボ}
\subsection{レオナ}
\begin{itembox}[l]{基本コンボ始動技}
\begin{tabular}{ll}
$\lbrack\ $近\ $\C\ \rbrack$\ or\ $\lbrack\ $近\ $\D\ \rbrack$\ \Cancel\
$\lbrack\ \rightarrow\ \B\ \rbrack$& $\rightarrow$\ (\rnum{1})\\
$\lbrack\ \downarrow\ \B\ \rbrack\ \times\ 1\ \sim\ 2$\ \Cancel\ $\lbrack\ $立\
$\B\ \rbrack$\ \Cancel\
$\lbrack\ \rightarrow\ \B\ \rbrack$&\ $\rightarrow$\ (\rnum{2})
\end{tabular}
\end{itembox}
\newpage
\subsection{ラルフ}
\begin{itembox}[l]{基本コンボ始動技}
\begin{tabular}{ll}
$\lbrack\ $近\ $\C\ (1)\ \rbrack$\ \Cancel\
$\lbrack\ \searrow\ \A\ \rbrack$& $\rightarrow$\ (\rnum{1})\\
$\lbrack\ $屈\ $\B\ \rbrack$\ \Cancel\
$\lbrack\ $屈\ $\A\ \rbrack$& $\rightarrow$\ (\rnum{2})\\
\end{tabular}
\end{itembox}
\newpage
\subsection{クラーク}
\begin{itembox}[l]{基本コンボ始動技}
\begin{tabular}{ll}
$\lbrack\ $近\ $\C\ (1)\ \rbrack$\ or\ $\lbrack\ $近\ $\D\ \rbrack$\Cancel\
$\lbrack\ \searrow\ \A\ \rbrack$& $\rightarrow$\ (\rnum{1})\\
$\lbrack\ $屈\ $\B\ \rbrack$\ \Cancel\
$\lbrack\ $屈\ $\A\ \rbrack$& $\rightarrow$\ (\rnum{2})\\
\end{tabular}
\end{itembox}
%%チーム名
\newpage
\part{女性格闘家チーム}
\section{コマンド}
\subsection{舞}
\begin{itembox}[l]{特殊技}
\begin{tabular}{lll}
浮羽&(空中で)\downarrow \ +\ \B&中段
\end{tabular}
\end{itembox}
\begin{itembox}[l]{必殺技}
\begin{tabular}{lll}
龍炎舞&\tatsu\ +\ \A\ or\ \C&EX版連続ガードじゃない\\
花蝶扇&\hado\ +\ \A\ or\ \C&EX版弾貫通\\%%備考%%
必殺忍蜂&\yoga\ +\ \B\ or\ \D&EX版最終弾前割り込み可\\%%備考%%
ムササビの舞(地上)&\vtame\ +\ \A\ or\ \C&1F空中判定\\
&&\A\ or\ \C追加で突進\\
ムササビの舞(空中)&(空中で)\tatsu\ +\ \A\ or\ \C&
\end{tabular}
\end{itembox}
\begin{itembox}[l]{超必殺技}
\begin{tabular}{lll}
超必殺忍蜂&\orochi \ +\ \B\ or\ \D&空中可
\end{tabular}
\end{itembox}
\begin{itembox}[l]{NEOMAX超必殺技}
\begin{tabular}{lll}
不知火流・くノ一の舞&\ryuko\ +\ \A\C&ヒット時のみ演出
\end{tabular}
\end{itembox}
\newpage
\subsection{キング}
\begin{itembox}[l]{特殊技}
\begin{tabular}{lll}
スライディング&$\searrow$\ +\ \B&下段%%備考%%
\end{tabular}
\end{itembox}
\begin{itembox}[l]{必殺技}
\begin{tabular}{lll}
ベノムストライク&\hado\ +\ \B\ or\ \D&空中可/EX版弾貫通\\%%備考%%
トラップショット&\syoryu\ +\ \B\ or\ \D&EX版無敵\\%%備考%%
トルネードキック'75&\gyakuyoga\ +\ \B\ or\ \D&EX版版無敵%%備考%%
\end{tabular}
\end{itembox}
\begin{itembox}[l]{超必殺技}
\begin{tabular}{lll}
サプライズローズ&\hado\hado\ +\ \A\ or\ \C&\\%%備考%%
ダブルストライク&\hado\hado\ +\ \B\ or\ \D&EX無し/弾貫通%%備考%%
\end{tabular}
\end{itembox}
\begin{itembox}[l]{NEOMAX超必殺技}
\begin{tabular}{lll}
ベノムショット&\tatsu\ +\ $\B\ \D$&%%備考%%
\end{tabular}
\end{itembox}
\newpage
\subsection{ユリ}
\begin{itembox}[l]{特殊技}
\begin{tabular}{lll}
ユリ雷神脚&(空中で)$\searrow$\ +\ \B&\\%%備考%%
燕翼&$\rightarrow$\ +\ \A&空中判定/中段
\end{tabular}
\end{itembox}
\begin{itembox}[l]{必殺技}
\begin{tabular}{lll}
砕破&\tatsu\ +\ \A\ or\ \C&弾判定\\%%備考%%
虎煌拳&\hado\ +\ \A\ or\ \C&EX版弾貫通\\%%備考%%
空牙&\syoryu\ +\ \A\ or\ \C&\\%%備考%%\\
(ユリちょうアッパー)&&\\
百烈ビンタ&\gyakuyoga\ +\ \B\ or\ \D&弱1F投げ/強移動投げ\\
&&EX版無敵打撃技\\%%備考%%
雷煌拳&\hado\ +\ \B\ or\ \D&弾判定/空中可\\%%備考%%
鳳翼&\syoryu\ +\ \B\ or\ \D&移動技/以下の技に派生\\%%備考%%
飛燕爪破&(鳳翼中)\ \A\ or\ \C&\\%%備考%%
ユリ雷神脚&(鳳翼中)\ \B\ or\ \D&\\%%備考%%
空中雷煌拳&(鳳翼中)\ \hado\ +\ \B\ or\ \D&弾判定\\%%備考%%
燕落とし&(鳳翼中)\A\C&なんでも判定%%備考%%
\end{tabular}
\end{itembox}
\begin{itembox}[l]{超必殺技}
\begin{tabular}{lll}
覇王翔吼拳&$\rightarrow$ \yoga\ +\ \A\ or\ \C&弾貫通/EX無し\\%%備考%%
飛燕鳳凰脚&\ryuko\ +\ \B\ or\ \D&%%備考%%
\end{tabular}
\end{itembox}
\begin{itembox}[l]{NEOMAX超必殺技}
\begin{tabular}{lll}
\end{tabular}
覇王雷煌拳&\tatsu\tatsu\ +\ \A\C&%%備考%%
\end{itembox}
\newpage
\section{コンボ}
\subsection{舞}
\begin{itembox}[l]{基本コンボ始動技}
\begin{tabular}{ll}
$\lbrack\ $近\ $\C\ \rbrack$\ or\ $\lbrack\ $近\ $\D\ \rbrack$& $\rightarrow$\
(\rnum{1})\\
$\lbrack\ $屈\ $\B\ \rbrack$\ \Cancel\
$\lbrack\ $屈\ $\A\ \rbrack$& $\rightarrow$\ (\rnum{2})
\end{tabular}
\end{itembox}
\newpage
\subsection{キング}
\begin{itembox}[l]{基本コンボ始動技}
\begin{tabular}{ll}
$\lbrack\ $近\ $\C\ \rbrack$\ or\ $\lbrack\ $近\ $\D\ \rbrack$\ \Cancel\ $\lbrack\
\searrow\ \D\ \rbrack$& $\rightarrow$\ (\rnum{1})\\
$\lbrack\ $屈\ $\B\ \rbrack$\ \times\ 1\ \sim\ 2\ \Cancel\
$\lbrack\ $立\ $\B\ \rbrack$\ \Cancel\ $\lbrack\
\searrow\ \D\ \rbrack$\ & $\rightarrow$\ (\rnum{2})
\end{tabular}
\end{itembox}
\newpage
\subsection{ユリ}
\subsubsection{中央コンボ}
\begingroup
 \renewcommand{\arraystretch}{1.2}
\begin{tabular*}{15.1cm}{@{\extracolsep{\fill}}|p{3em}||p{12.9cm}|}\hline
\multicolumn{2}{|p{14.6cm}|}{
\PG{0}\ \ \ \DG{0}
}\\\bhline{2pt}
コンボ&
\command{近\C}\ \Cancel\ \command{\syoryu\ +\ \C}
%%コンボルート
\\\hline
補足&基本コンボ。
%%補足
\\\bhline{2pt}%%\hline\hline
コンボ&
\command{$\downarrow$\B}\ $\times\ 1\ \sim\ 2$\ \Cancel\ \command{立\B}\
\Cancel\
\command{\syoryu\ +\ \C}
%%コンボルート
\\\hline
補足&小技始動基本コンボ。特に断らない限り以下こちらの始動も可。
%%補足
\\\hline\hline\multicolumn{2}{|p{14.6cm}|}{
\PG{0}\ \ \ \DG{50}
}\\\bhline{2pt}
コンボ&
\command{近\C}\ \Cancel\ \command{\syoryu\ +\ \C(1)}\ \DC\ \command{\syoryu\ +\
\D}\ \Cancel\ \command{\A\ or\ \C}\ \migi\ \command{$\downarrow$\C}\
\Cancel\ \command{\syoryu\ +\ \D}\ \Cancel\ \command{\A\C}\
%%コンボルート
\\\hline
補足&位置が入れ替わる。
%%補足
\\\bhline{2pt}%%\hline\hline
コンボ&
\command{\gyakuyoga\ +\ \B\ or\ \D} \DC\ \command{\syoryu\ +\
\D}\ \Cancel\ \command{\A\ or\ \C}\ \migi\ \command{$\downarrow$\C}\
\Cancel\ \command{\syoryu\ +\ \D}\ \Cancel\ \command{\A\C}\
%%コンボルート
\\\hline
補足&コマンド投げ始動技。位置が入れ替わる。
%%補足
\\\hline\hline
\multicolumn{2}{|p{14.6cm}|}{
\PG{1}\ \ \ \DG{0}
}\\\bhline{2pt}
コンボ&
\command{近\C}\ \Cancel\ \command{\ryuko\ +\ \B\ or\ \D}
%%コンボルート
\\\hline
補足&もう少し減るものがあるが、とりあえず簡単。
%%補足
\\\bhline{2pt}%%\hline\hline
コンボ&
\command{近\C}\ \Cancel\ \command{\tatsu\ +\ \A\C}\ \migi\ \command{\syoryu\ +\
\D}\ \Cancel\ \command{\A\ or\ \C}\ \migi\ \command{$\downarrow$\C}\
\Cancel\ \command{\syoryu\ +\ \D}\ \Cancel\ \command{\A\C}\
%%コンボルート
\\\hline
補足&位置が入れ替わる。小技始動不可。
%%補足
\\\hline\hline
\multicolumn{2}{|p{14.6cm}|}{
\PG{1}\ \ \ \DG{50}
}\\\bhline{2pt}
コンボ&
\command{近\C}\ \Cancel\ \command{\syoryu\ +\ \C}\ \SC\ \command{\ryuko\ +\ \B\ or\ \D}
%%コンボルート
\\\hline
補足&\command{\ryuko\ +\ \B\ or\ \D}は\command{\gyakuyoga\ +\ \B\ or\ \D}でOK。
%%補足
\\\bhline{2pt}%%\hline\hline
コンボ&
\command{\gyakuyoga\ +\ \B\ or\ \D}\ \SC\ \command{\ryuko\ +\ \B\ or\ \D}
%%コンボルート
\\\hline
補足&コマ投げ始動。
%%補足
\\\hline\hline
\multicolumn{2}{|p{14.6cm}|}{
\PG{2}\ \ \ \DG{0}}
\\\bhline{2pt}
コンボ&
\command{近\C}\ \Cancel\ \command{\ryuko\ +\ \B\D}
%%コンボルート
\\\hline
補足&もう少し減るものがあるが、とりあえず簡単。
%%補足
\\\bhline{2pt}%%\hline\hline
コンボ&
\command{近\C}\ \Cancel\ \command{\tatsu\ +\ \A\C}\ \migi\ \command{\syoryu\ +\
\D}\ \Cancel\ \command{\A\ or\ \C}\ \migi\ \command{\ryuko\ +\ \B\ or\ \D}
%%コンボルート
\\\hline
補足&小技始動不可。
%%補足
\\\hline\hline
\multicolumn{2}{|p{14.6cm}|}{
\PG{2}\ \ \ \DG{50}
}\\\bhline{2pt}
コンボ&
\command{近\C}\ \Cancel\ \command{\syoryu\ +\ \C}\ \SC\ \command{\ryuko\ +\ \B\D}
%%コンボルート
\\\hline
補足&\command{\ryuko\ +\ \B\ or\ \D}は\command{\gyakuyoga\ +\ \B\D}でOK。
%%補足
\\\bhline{2pt}%%\hline\hline
コンボ&
\command{\gyakuyoga\ +\ \B\ or\ \D}\ \SC\ \command{\ryuko\ +\ \B\D}
%%コンボルート
\\\hline
補足&コマ投げ始動。
%%補足
\\\bhline{2pt}
\end{tabular*}
\endgroup
\newpage
\subsubsection{端コンボ}
\begingroup
 \renewcommand{\arraystretch}{1.2}
\begin{tabular*}{15.1cm}{@{\extracolsep{\fill}}|p{3em}||p{12.9cm}|}\hline
\multicolumn{2}{|p{14.6cm}|}{
\PG{0}\ \ \ \DG{50}
}\\\bhline{2pt}
コンボ&
\command{近\C}\ \Cancel\ \command{\syoryu\ +\ \C}\ \DC\ \command{\tatsu\ +\ \C}\\
&\migi\ \command{\tatsu\ +\ \A}\ \migi\ \command{$\downarrow$\C}\
\Cancel\ \command{\syoryu\ +\ \D}\ \Cancel\ \command{\A\C}\
%%コンボルート
\\\hline
補足&位置が入れ替わる。
%%補足
\\\bhline{2pt}
コンボ&
\command{\gyakuyoga\ +\ \B\ or\ \D} \DC\ \command{\tatsu\ +\ \C}\\
&\migi\ \command{\tatsu\ +\ \A}\ \migi\ \command{$\downarrow$\C}\
\Cancel\ \command{\syoryu\ +\ \D}\ \Cancel\ \command{\A\C}\
%%コンボルート
\\\hline
補足&コマ投げ始動。位置が入れ替わる。
%%補足
\\\hline\hline
\multicolumn{2}{|p{14.6cm}|}{
\PG{1}\ \ \ \DG{0}
}\\\bhline{2pt}
コンボ&
\command{近\C}\ \Cancel\ \command{\tatsu\ +\ \A\C}\ \migi\ \command{\tatsu\ +\
\C}\\
& \migi\ \command{\hado\ +\ \C}\ \migi \command{$\downarrow$\C}\
\Cancel\ \command{\syoryu\ +\ \D}\ \Cancel\ \command{\A\C}
%%コンボルート
\\\hline
補足&小技始動不可。位置が入れ替わる。
%%補足
\\\hline\hline
\multicolumn{2}{|p{14.6cm}|}{
\PG{1}\ \ \ \DG{50}
}\\\bhline{2pt}
コンボ&
\command{近\C}\ \Cancel\ \command{\tatsu\ +\ \A\C}\ \migi\ \command{\tatsu\ +\
\C}\\
& \migi\ \command{\hado\ +\ \C}\ \migi\ \command{\syoryu\ +\ \C(1)}\ \DC\
\command{\syoryu\ +\ \D}\ \Cancel\ \command{\A\ or\ \C}\ \migi\ \command{$\downarrow$\C}\
\Cancel\ \command{\syoryu\ +\ \D}\ \Cancel\ \command{\A\C}\
%%コンボルート
\\\hline
補足&小技始動不可。位置が入れ替わる。
%%補足
\\\hline\hline
\multicolumn{2}{|p{14.6cm}|}{
\PG{2}\ \ \ \DG{0}
}\\\bhline{2pt}
コンボ&
\command{近\C}\ \Cancel\ \command{\tatsu\ +\ \A\C}\ \migi\ \command{\tatsu\ +\
\C}\\
& \migi\ \command{\hado\ +\ \C}\ \migi\ \command{\ryuko\ +\ \B\ or\ \D}
%%コンボルート
\\\hline
補足&小技始動不可
%%補足
\\\hline\hline
\multicolumn{2}{|p{14.6cm}|}{
\PG{2}\ \ \ \DG{50}
}\\\bhline{2pt}
コンボ&\command{近\C}\ \Cancel\ \command{\tatsu\ +\ \A\C}\ \migi\ \command{\tatsu\ +\
\C}\\
& \migi\ \command{\hado\ +\ \C}\ \migi\ \command{\syoryu\ +\ \C(1)}\ \DC\
\command{\tatsu\ +\ \C}\ \migi\ \command{\hado\ +\ \C}\ \migi\ \command{\ryuko\ +\ \B\ or\ \D}
%%コンボルート
\\\hline
補足&小技始動不可
%%補足
\\\bhline{2pt}%%\hline\hline
コンボ&
\command{近\C}\ \Cancel\ \command{\syoryu\ +\ \C}\ \DC\ \command{\tatsu\ +\ \C}\\
&\migi\ \command{\tatsu\ +\ \A}\ \migi\ \command{\ryuko\ +\ \B\D}
%%コンボルート
\\\hline
補足&特になし。
%%補足
\\\bhline{2pt}%%\hline\hline
コンボ&
\command{\gyakuyoga\ +\ \B\ or\ \D}\ \DC\ \command{\tatsu\ +\ \C}\\
&\migi\ \command{\tatsu\ +\ \A}\ \migi\ \command{\ryuko\ +\ \B\D}
%%コンボルート
\\\hline
補足&コマ投げ始動。
%%補足
\\\bhline{2pt}%%\hline\hline
\end{tabular*}
\endgroup
%%1チームここまで
\newpage
\section{コマンド}
\subsection{爪庵}
\begin{itembox}[l]{特殊技}
\begin{tabular}{lll}
外式・夢弾&$\rightarrow$\ +\ \A&\A\ 追加で二段目発生\\
&&初段ヒットバック小\\%%備考%%
外式・杭&$\searrow$\ +\ \C&中段\\%%備考%%
外式・百合折り&(空中で)$\leftarrow$\ +\ \B&バクステ中可%%備考%%
\end{tabular}
\end{itembox}
\begin{itembox}[l]{必殺技}
\begin{tabular}{lll}
百四式・鵺討ち&\syoryu\ +\ \A\ or\ \C&弱上半身無敵\\%%備考%%
四百壱式・衝月&\tatsu\ +\ \A\ or\ \C&下段無敵\\ %%備考%%
百弐拾九式・明烏&\tatsu\ +\ \B\ or\ \D&相手付近で攻撃発生/EX版弾無敵\\ %%備考%%
二百参式・槌椿&\yoga\ +\ \A\ or\ \C&投技/EX版発生まで無敵%%備考%%
\end{tabular}
\end{itembox}
\begin{itembox}[l]{超必殺技}
\begin{tabular}{lll}
禁千弐百十壱式・八稚女&\ryuko\ +\ \A\ or\ \C&強制ダウン\\ %%備考%%
\end{tabular}
\end{itembox}
\begin{itembox}[l]{NEOMAX超必殺技}
\begin{tabular}{lll}
禁千弐百十八式・八咫烏&\hado\hado\ +\ $\B\ \D$&初段ヒットか否かで演出変化%%備考%%
\end{tabular}
\end{itembox}
\newpage
\subsection{マチュア}
\begin{itembox}[l]{必殺技}
\begin{tabular}{lll}
デスペアー&\syoryu\ +\ \A\ or\ \C&\A\ or\ \C追加で攻撃発生\\%%備考%%
デスロウ&\tatsu\ +\ \A\ or\ \C&通常3回/EX5回派生\\%%備考%%
メタルマサカー&\tatsu\ +\ \B\ or\ \D&相手の近くで攻撃発生\\%%備考%%
エボニーティアーズ&\ryuko +\ \A\ or\ \C&EX版弾貫通%備考%%
\end{tabular}
\end{itembox}
\begin{itembox}[l]{超必殺技}
\begin{tabular}{lll}
ヘブンズゲイト&\orochi\ +\ \B\ or\ \D&画面端まで運ぶ\\
&&EX版無し\\ %%備考%%
ノクターナルライツ&\hado\hado\ +\ \A\ or\ \C&%%備考%%
\end{tabular}
\end{itembox}
\begin{itembox}[l]{NEOMAX超必殺技}
\begin{tabular}{lll}
アウェイキングブラッド&\orochi\ +\ \A\C&中段%%備考%%
\end{tabular}
\end{itembox}
\newpage
\subsection{バイス}
\begin{itembox}[l]{特殊技}
\begin{tabular}{lll}
ドッケン&$\rightarrow$\ +\ \A&中段\\%%備考%%
ターゲットコンボ?&近\D\ \Cancel\ \D&%%備考%%
\end{tabular}
\end{itembox}
\begin{itembox}[l]{必殺技}
\begin{tabular}{lll}
ディーサイド&\yoga\ +\ \B\ or\ \D&強版ヒット時キャンセル可\\
&&EX版なんでも判定\\%%備考%%
ゴアフェスト&\tenti\ +\ \A\ or\ \C&1F投げ\\%%備考%%
スプラッシュ&\syoryu\ +\ \A\ or\ \C&対地移動投げ\\%%備考%%
メイヘム&\tatsu\ +\ \A\ or\ \C&EX版無敵\\%%備考%%
スプラッシュ(派生)&(メイヘム中)\hado\ +\ \A\ or\ \C&メイヘムから連続ヒット%%備考%%
\end{tabular}
\end{itembox}
\begin{itembox}[l]{超必殺技}
\begin{tabular}{lll}
ネガティブゲイン&\gyakuyoga\gyakuyoga\ +\ \B\ or\ \D&1F投げ\\%%備考%%
オーバーキル&(空中で)&対空投げ \\%%備考%%
&$\swarrow\ \downarrow\ \searrow\ \rightarrow\ \nearrow\ \uparrow\
\downarrow$\ +\ \A\ or\ \C&EX版なし
\end{tabular}
\end{itembox}
\begin{itembox}[l]{NEOMAX超必殺技}
\begin{tabular}{lll}
アウェイキングブラッド&\orochi\ +\ \A\C&%%備考%%
\end{tabular}
\end{itembox}
\newpage
\section{コンボ}
\subsection{爪庵}
\newpage
\subsection{マチュア}
\subsubsection{中央・端コンボ}
\begingroup
 \renewcommand{\arraystretch}{1.2}
\begin{tabular*}{15.1cm}{@{\extracolsep{\fill}}|p{3em}||p{12.9cm}|}\hline
\multicolumn{2}{|p{14.6cm}|}{
\PG{0}\ \ \ \DG{0}
}\\\bhline{2pt}
コンボ&
\command{近\C}\ \Cancel\ \command{\tatsu\ +\ \A\ or\ \C}\ $\times\ 3$
%%コンボルート
\\\hline
補足&
基礎コンボ。
%%補足
\\\hline
コンボ&
\command{$\downarrow$\ \B}\ $\times\ 1\ \sim\ 2$\ \Cancel\ \command{立\A}\ or\
\command{$\downarrow$\ \A}\ \Cancel\ \command{\tatsu\ +\ \A\ or\ \C}\ $\times\
3$
%%コンボルート
\\\hline
補足&
小技始動基礎コンボ。以下のコンボはこちらの始動でも可。
%%補足
\\\hline\hline
\multicolumn{2}{|p{14.6cm}|}{
\PG{0}\ \ \ \DG{50}
}\\\bhline{2pt}
コンボ&\command{近\C}\ \Cancel\ \command{\tatsu\ +\ \A\ or\ \C}\ $\times\ 3$\ \DC\
\command{\tatsu\ +\ \D}\\
& \migi\ \command{\tatsu\ +\ \B}\ $\times\ 2$\ \migi\ 
(\command{近\C}\ \Cancel\ \command{\tatsu\ +\ \B})
%%コンボルート
\\\hline
補足&最後の\command{近\C}は端到達時の追撃し、\command{\tatsu\ +\ \B}で隙をフォロー。\\
&\command{\tatsu\ +\ \A\ or\ \C}\ $\times\ 3$\ \DC\
\command{\tatsu\ +\ \D}のタイミングは慣れが必要。ダメージが高く、ゲージ回収も優秀。
%%補足
\\\hline\hline
\multicolumn{2}{|p{14.6cm}|}{
\PG{1}\ \ \ \DG{0}
}\\\bhline{2pt}
コンボ&\command{近\C}\ \Cancel\ \command{\hado\ \hado\ +\ \A\ or\ \C}\ 
%%コンボルート
\\\hline
補足&ドライブを使わない1ゲージで簡単な小技始動も可能なコンボ。小技始動なら分割が効いてより簡単。ダメージは微妙なので倒しきれる時用。
%%補足
\\\hline\hline
\multicolumn{2}{|p{14.6cm}|}{
\PG{1}\ \ \ \DG{50}
}\\\bhline{2pt}
コンボ&\command{近\C}\ \Cancel\ \command{\tatsu\ +\ \A\C}\ $\times\ 5$\ \DC\
\command{\tatsu\ +\ D}\\
& \migi\ \command{\tatsu\ +\ \B}\ $\times\ 2$\ \migi\ 
(\command{近\C}\ \Cancel\ \command{\tatsu\ +\ \B})
%%コンボルート
\\\hline
補足&最後に関しては同上。簡単で高ダメージ、良回収。
%%補足
\\\hline\hline
\multicolumn{2}{|p{14.6cm}|}{
\PG{2}\ \ \ \DG{0}
}\\\bhline{2pt}
コンボ&\command{近\C}\ \Cancel\ \command{\hado\ \hado\ +\ \A\C}\ 
%%コンボルート
\\\hline
補足&1ゲージの時と同様、ダメージはゲージ使用量の割に微妙なのでドライブゲージを使わずに倒しきれる時用。
%%補足
\\\hline\hline
\end{tabular*}
\endgroup

\newpage
\subsection{バイス}
\subsubsection{中央・端コンボ}\begingroup
 \renewcommand{\arraystretch}{1.2}
\begin{tabular*}{15.1cm}{@{\extracolsep{\fill}}|p{3em}||p{12.9cm}|}\hline
\multicolumn{2}{|p{14.6cm}|}{
\PG{0}\ \ \ \DG{0}
}\\\bhline{2pt}
コンボ&
\command{近\D\ or\ 近\C}\ \Cancel\ \command{$\rightarrow$\ \A}\\
& \Cancel\ \command{\tatsu\ +\ \C}\ \Cancel\ \command{\hado\ +\ \A\ or\ \C}
%%コンボルート
\\\hline
補足&
ダメージ重視基礎コンボ。
%%補足
\\\bhline{2pt}%%\hline\hline
コンボ&
\command{近\D\ or\ 近\C}\ \Cancel\ \command{$\rightarrow$\ \A}\\
& \Cancel\ \command{\yoga\ +\ \B}\ \migi\ \command{\tenti\ +\ \A\ or\ \C}
%%コンボルート
\\\hline
補足&
状況重視基礎コンボ。相手にあまりゲージを与えない。
%%補足
\\\bhline{2pt}%%\hline\hline
コンボ&
\command{$\downarrow$\ \B}\ $\times\ 1\ \sim\ 3$\ \Cancel\ \command{\tatsu\ +\ 
\A }\ \Cancel\ \command{\hado\ +\ \A\ or\ \C}
%%コンボルート
\\\hline
補足&
小技始動基本コンボ。最後の\command{$\downarrow$\ \B}は\command{立 \B}でもいい。
%%補足
\\\hline\hline
\multicolumn{2}{|p{14.6cm}|}{
\PG{0}\ \ \ \DG{50}
}\\\bhline{2pt}
コンボ
&\command{近\D\ or\ 近\C}\ \Cancel\ \command{$\rightarrow$\ \A}\ \Cancel\
\command{\tatsu\ +\ \C}\ \\
&\DC\ \command{\yoga\ +\ \D}\\
& \Cancel\ \command{\tatsu\ +\ \C}\ \Cancel\
\command{\hado\ +\ \A\ or\ \C}
%%コンボルート
\\\hline
補足&
ドライブゲージをパワーゲージに還元できる。
%%補足
\\\bhline{2pt}%%\hline\hline
コンボ&
\command{$\downarrow$\ \B}\ $\times\ 1\ \sim\ 3$\ \Cancel\ \command{\tatsu\ +\ 
\A }\ \DC\ \command{\yoga\ +\ \D}\\
& \Cancel\ \command{\tatsu\ +\ \C}\ \Cancel\
\command{\hado\ +\ \A\ or\ \C}
%%コンボルート
\\\hline
補足&
小技始動からダメージを増やしつつドライブゲージをパワーゲージに還元。
%%補足
\\\hline\hline
\multicolumn{2}{|p{14.6cm}|}{
\PG{1}\ \ \ \DG{0}
}\\\bhline{2pt}
コンボ&
\command{近\D\ or\ \C}\ \Cancel\ \command{$\rightarrow$\ \A}\ \Cancel\
\command{\tatsu\ + \ \A\C}\\
&\migi\ \command{\tatsu\ +\ \A}\ \Cancel\ \command{\hado\ +\ \A\ or\ \C}
%%コンボルート
\\\hline
補足&
\command{$\downarrow$\ \B}\ $\times\ 1\ \sim\ 2$での始動も可能だが、距離次第でつながらない。
%%補足
\\\bhline{2pt}
コンボ&
\command{$\downarrow$\ \B}\ $\times\ 1\ \sim\ 3$\ \Cancel\ \command{\yoga\ +\
\B\D}\ \migi\ \command{近\D}\\
& \Cancel\ \command{$\rightarrow$\ \A}\Cancel\
\command{\tatsu\ +\ \C}\ \Cancel\ \command{\hado\ +\ \A\ or\ \C}%%コンボルート
\\\hline
補足&
\command{\yoga\ +\ \B\D}からの目押しはそこまで難しくない。\\
&\command{近\D}のほうが発生が早いので簡単。
%%補足
\\\hline\hline
\multicolumn{2}{|p{14.6cm}|}{
\PG{1}\ \ \ \DG{50}
}\\\bhline{2pt}
コンボ&
\command{近\D\ or\ \C}\ \Cancel\ \command{$\rightarrow$\ \A}\ \Cancel\
\command{\tatsu\ +\ \C}\\
& \DC\ \command{\yoga\ +\ \B\D}\ \migi\ \command{近\D}\
\Cancel\ \command{$\rightarrow$\ \A}\\
&\Cancel\ \command{\tatsu\ +\ \C}\ \Cancel\ \command{\hado\ +\ \A\ or\ \C}
\\\hline%%コンボルート
補足&特になし
%%補足
\\\bhline{2pt}%%\hline\hline
コンボ&
\command{$\downarrow$\ \B}\ $\times\ 1\ \sim\ 3$\ \Cancel\ \command{\tatsu\ +\ 
\A }\\
& \DC\ \command{\yoga\ +\ \B\D}\ \migi\ \command{近\D}\
\Cancel\ \command{$\rightarrow$\ \A}\\
&\Cancel\ \command{\tatsu\ +\ \C}\ \Cancel\ \command{\hado\ +\ \A\ or\ \C}
%%コンボルート
\\\hline
補足&
特になし
%%補足
\\\bhline{2pt}%%\hline\hline
\end{tabular*}
\endgroup
\newpage
\begingroup
\renewcommand{\arraystretch}{1.2}
\begin{tabular*}{15.1cm}{@{\extracolsep{\fill}}|p{3em}||p{12.9cm}|}\hline
\multicolumn{2}{|p{14.6cm}|}{
\PG{2}\ \ \ \DG{0}
}\\\bhline{2pt}
コンボ&
\command{近\D\ or\ \C}\ \Cancel\ \command{$\rightarrow$\ \A}\ \Cancel\
\command{\tatsu\ + \ \A\C}\\
&\migi\ \command{近\D}\ \migi\ \command{\yoga\ +\ \B\D}\ \migi\ \command{近\D}\
\Cancel\ \command{$\rightarrow$\ \A}\\
&\Cancel\ \command{\tatsu\ +\ \C}\ \Cancel\ \command{\hado\ +\ \A\ or\ \C}
%%コンボルート
\\\hline
補足&
\command{近\D}\ \migi\ \command{\yoga\ +\ \B\D}の代わりに\\
&\command{$\downarrow$\ \D}\
\Cancel\ \command{\yoga\ +\ \B\D}\ や\\
&\command{小J\C\D}\ \migi\ \command{\yoga\
+\ \B\D}\ のほうが高いが難しくなる
%%補足
\\\hline\hline
\multicolumn{2}{|p{14.6cm}|}{
\PG{2}\ \ \ \DG{50}
}\\\bhline{2pt}
コンボ&
\command{近\D\ or\ \C}\ \Cancel\ \command{$\rightarrow$\ \A}\ \Cancel\
\command{\tatsu\ + \ \A\C}\\
&\migi\ \command{近\D}\ \migi\ \command{\yoga\ +\ \B\D}\ \migi\ \command{近\D}\
\Cancel\ \command{$\rightarrow$\ \A}\\
&\Cancel\ \command{\tatsu\ +\ \C}\ \Cancel\ \DC\
\command{\yoga\ +\ \D}\\
& \Cancel\ \command{\tatsu\ +\ \C}\ \Cancel\
\command{\hado\ +\ \A\ or\ \C}
%%コンボルート
\\\hline
補足&
特になし
%%補足
\\\bhline{2pt}
\end{tabular*}
\endgroup
\newpage
\part{龍虎チーム}%%チーム名
\section{コマンド}
\subsection{リョウ}
\begin{itembox}[l]{特殊技}
\begin{tabular}{lll}
氷柱割り&$\rightarrow$\ +\ \A&中段\\%%備考%%
上段受け&$\rightarrow$\ +\ \B&受け成立時キャンセル可\\%%備考%%
下段受け&$\searrow$\ +\ \B&受け成立時キャンセル可%%備考%%
\end{tabular}
\end{itembox}
\begin{itembox}[l]{必殺技}
\begin{tabular}{lll}
虎煌拳&\hado\ +\ \A\ or\ \C&強/EX版弾貫通\\%%備考%%
虎砲&\syoryu\ +\ \A\ or\ \C&弱上半身無敵/強全身無敵\\%%備考%%
飛燕疾風脚&\gyakuyoga\ +\ \B\ or\ \D&\\%%備考%%
暫烈拳&$\rightarrow\ \leftarrow\ \rightarrow$\ +\ \A\ or\ \C&%%備考%%
\end{tabular}
\end{itembox}
\begin{itembox}[l]{超必殺技}
\begin{tabular}{lll}
覇王翔吼拳&$\rightarrow$\yoga\ +\ \A\ or\ \C&弾貫通/EX版無し\\%%備考%%
龍虎乱舞&\ryuko\ +\ \A\ or\ \C&%%備考%%
\end{tabular}
\end{itembox}
\begin{itembox}[l]{NEOMAX超必殺技}
\begin{tabular}{lll}
真・天地覇煌拳&\hado\hado\ +\ \A\C&ガードクラッシュ値100%%備考%%
\end{tabular}
\end{itembox}
\newpage
\subsection{ロバート}
\begin{itembox}[l]{特殊技}
\begin{tabular}{lll}
アッパー&$\rightarrow$\ +\ \A&キャンセル時二段技\\%%備考%%
ローキック&$\rightarrow$\ +\ \B&下段\\%%備考%%
後ろ蹴り&(空中で)$\B\ \D$&バクステ中可%%備考%%
\end{tabular}
\end{itembox}
\begin{itembox}[l]{必殺技}
\begin{tabular}{lll}
龍撃拳&\hado\ +\ \A\ or\ \C&EX版弾貫通\\%%備考%%
龍牙&\syoryu\ +\ \A\ or\ \C&弱上半身無敵/強全身無敵\\%%備考%%
飛燕疾風脚&\hien\ +\ \B\ or\ \D&\\%%備考%%
飛燕龍神脚&(空中で)\tatsu\ +\ \B\ or\ \D&\\%%備考%%
幻影脚&$\rightarrow\ \leftarrow\ \rightarrow$\ +\ \B\ or\ \D&\\%%備考%%
龍連・幻影脚&\yoga\ +\ \B\ or\ \D&投げ技%%備考%%
\end{tabular}
\end{itembox}
\begin{itembox}[l]{超必殺技}
\begin{tabular}{lll}
覇王翔吼拳&$\rightarrow$\yoga\ +\ \A\ or\ \C&弾貫通/EX版無し\\%%備考%%
龍虎乱舞&\ryuko\ +\ \A\ or\ \C&%%備考%%
\end{tabular}
\end{itembox}
\begin{itembox}[l]{NEOMAX超必殺技}
\begin{tabular}{lll}
飛燕疾風龍神脚&\ryuko\ +\ $\B\ \D$&空中可%%備考%%
\end{tabular}
\end{itembox}
\newpage
\subsection{タクマ}
\begin{itembox}[l]{特殊技}
\begin{tabular}{lll}
足刀蹴り&$\rightarrow$\ +\ \B&%%備考%%
\end{tabular}
\end{itembox}
\begin{itembox}[l]{必殺技}
\begin{tabular}{lll}
虎煌拳&\hado\ +\ \A\ or\ \C&EX版相手サーチ\\%%備考%%
極限虎煌&\hado\ +\ \B\ or\ \D&強版ヒット時自動で移動\\%%備考%%
飛燕疾風脚&\hien\ +\ \B\ or\ \D&\\%%備考%%
暫烈拳&$\rightarrow\ \leftarrow\ \rightarrow$\ +\ \A\ or\ \C&EX版なんでも判定\\%%備考%%
極限崩撃&\gyakuyoga\ +\ \A\ or\ \C&投げ技/EX版発生まで無敵%%備考%%
\end{tabular}
\end{itembox}
\begin{itembox}[l]{超必殺技}
\begin{tabular}{lll}
覇王至高権&$\rightarrow$\yoga\ +\ \A\ or\ \C&弾貫通/EX版弾貫通\times 3\\%%備考%%
龍虎乱舞&\ryuko\ +\ \A\ or\ \C&%%備考%%
\end{tabular}
\end{itembox}
\begin{itembox}[l]{NEOMAX超必殺技}
\begin{tabular}{lll}
毘瑠斗圧覇&\hado\hado\ +\ \A\ or\ \C&下降部分相手サーチ%%備考%%
\end{tabular}
\end{itembox}
\newpage
\section{コンボ}
\subsection{リョウ}
\subsection{ロバート}
\subsection{タクマ}
\newpage
\part{キムチーム}%%チーム名
\section{コマンド}
\subsection{キム}
\begin{itembox}[l]{特殊技}
\begin{tabular}{lll}
ネリチャギ&$\rightarrow$\ +\ \B&中段\\%%備考%%
トラ・ヨプチャギ&$\rightarrow\ \rightarrow$\ +\ \A&先行入力が効く/空中判定%%備考%%
\end{tabular}
\end{itembox}
\begin{itembox}[l]{必殺技}
\begin{tabular}{lll}
半月斬&\tatsu\ +\ \B\ or\ \D&\\%%備考%%
空中半月斬&\tatsy\ +\ \B\ or\ \D&強版中段/EX版無敵\\%%備考%%
飛翔脚&(空中で)\hado\ +\ \B\ or\ \D&弱/強版中\Dで追加攻撃\\%%備考%%
飛燕斬&\vtame\ +\ \B\ or\ \D&弱上半身無敵/強完全無敵\\%%備考%%
&&強版中\ $\downarrow\ \D$で追加攻撃
\end{tabular}
\end{itembox}
\begin{itembox}[l]{超必殺技}
\begin{tabular}{lll}
鳳凰脚&\tatsu\ $\swarrow\ \rightarrow$\ +\ \B\ or\ \D&空中下\\%%備考%%
鳳凰飛天脚&\hado\ \hado\ +\ \B\ or\ \D&ガード時有利/EX版無し%%備考%%
\end{tabular}
\end{itembox}
\begin{itembox}[l]{NEOMAX超必殺技}
\begin{tabular}{lll}
零式鳳凰脚&\tatsu\ $\swarrow\ \rightarrow$+\ \A\C&%%備考%%
\end{tabular}
\end{itembox}
\newpage
\subsection{ホア}
\begin{itembox}[l]{特殊技}
\begin{tabular}{lll}
スライディング&$\searrow$\ +\ \B&下段%%備考%%
\end{tabular}
\end{itembox}
\begin{itembox}[l]{必殺技}
\begin{tabular}{lll}
ドラゴンキック&\syoryu\ +\ \B\ or\ \D&EX版なんでも判定/空中可\\%%備考%%
&&EX版方向キー\ +\ \B\ or\ \D で追加攻撃\\
ドラゴンテイル&\tatsu\ +\ \B\ or\ \D&空中可/地上EX版弾無敵\\%%備考%%
爆裂拳&\hado\ +\ \A\ or\ \C&通常版同コマンドでフィニッシュ\\%%備考%%
\end{tabular}
\end{itembox}
\begin{itembox}[l]{超必殺技}
\begin{tabular}{lll}
ドリンク飲み&\tatsu\ \tatsu\ +\ \A\ or\ \C&EX版無し\\
&&一定時間強化\\
&&防御力低下\\%%備考%%
ドラゴンダンス&\hado\ \hado\ +\ \B\ or\ \D&\\%%備考%%
ドラゴンバックブリーカー&\ryuko\ +\ \A\ or\ \C&弱1F投げ\\
&&強移動投げ\\%%備考%%
&&EX版無し
\end{tabular}
\end{itembox}
\begin{itembox}[l]{NEOMAX超必殺技}
\begin{tabular}{lll}
ファイナル・ドラゴンキック&\ryuko\ +\ $\B\ \D$&スカヒット時追撃可%%備考%%
\end{tabular}
\end{itembox}
\newpage
\subsection{ライデン}
\begin{itembox}[l]{必殺技}
\begin{tabular}{lll}
毒霧&\tatsu\ +\ \A\ or\ \C&弾判定\\%%備考%%
ジャイアントボム&\hien\ +\ \A\ or\ \C&通常版発生までに\\
&&$\A\ \B$入力でフェイント\\%%備考%%
ヘッドクラッシュ&\yoga\ +\ \B\ or\ \D&1F投げ\\%%備考%%
スーパードロップキック&\B\ or\ \D を4秒以上&すべての地上技から\\%%備考%
&押して離す&キャンセルできる\\
&&押下時間に応じて性能変化
\end{tabular}
\end{itembox}
\begin{itembox}[l]{超必殺技}
\begin{tabular}{lll}
スーパーライデンドロップ&\gyakuyoga\ $\times\ 2$ +\ \A\ or\ \C&1F投げ\\%%備考%%
クレイジートレイン&\hado\ \hado\ +\ \A\ or\ \C&EX版無し%%備考%%
\end{tabular}
\end{itembox}
\begin{itembox}[l]{NEOMAX超必殺技}
\begin{tabular}{lll}
ライデンボンバー&\hado\ \hado\ +\ $\B\ \D$&%%備考%%
\end{tabular}
\end{itembox}
\newpage
\section{コンボ}
\subsection{キム}
\subsubsection{中央コンボ}
\begingroup
 \renewcommand{\arraystretch}{1.2}
\begin{tabular*}{15.1cm}{@{\extracolsep{\fill}}|p{3em}||p{12.9cm}|}\hline
\multicolumn{2}{|p{14.6cm}|}{
\PG{0}\ \ \ \DG{0}
}\\\bhline{2pt}
コンボ&
\command{$\downarrow$ \B}$\times\ 1\sim\ 3$\ \Cancel\ \command{立\B}\ \Cancel\
\command{$\rightarrow\ \rightarrow$\ +\ \A$}\ \Cancel\ \command{\tatsu\ +\ \B}
%%コンボルー
\\\hline
補足&
基本コンボ。強制ダウンを奪える。\command{立\B}から\command{$\rightarrow\ \rightarrow$\ +\
\A$}が少し難しい。
%%補足
\\\hline\hline
\multicolumn{2}{|p{14.6cm}|}{
\PG{0}\ \ \ \DG{50}
}\\\bhline{2pt}
コンボ&
\command{近\C\ or\ 近\D}\ \Cancel\ \command{\tatsu\ +\ \D}\\
& \DC\ \command{\vtame\
+\ \D}\ \Cancel\ \command{$\downarrow$\ +\ \D}
%%コンボルート
\\\hline
補足&
簡単でそこそこ減って強制ダウン。ゲージに余裕が有るときに。
%%補足
\\\hline\hline
\multicolumn{2}{|p{14.6cm}|}{
\PG{1}\ \ \ \DG{0}
}\\\bhline{2pt}
コンボ&
\command{$\downarrow$ \B}$\times\ 1\sim\ 3$\ \Cancel\ \command{立\B}\ \Cancel\
\command{\tatsu\ +\ \B\D}\\
& \migi\ \command{$\rightarrow\ \rightarrow$\ +\ \A$}\
\Cancel\ \command{\tatsu\ +\ \D}
%%コンボルート
\\\hline
補足&
かなり運んで強制ダウン。\command{$\rightarrow\ \rightarrow$\ +\ \A$}は先行入力が効く。
%%補足
\\\hline\hline
\multicolumn{2}{|p{14.6cm}|}{
\PG{1}\ \ \ \DG{50}
}\\\bhline{2pt}
コンボ&\command{近\C\ or\ 近\D}\ \Cancel\ \command{\tatsu\ +\ \D}\ \DC\
\command{\vtame\ +\ \B\D}\
%%コンボルート
\\\hline
補足&
強制ダウン。倒しきれる時用。%%補足
\\\bhline{2pt}%%\hline\hline
\end{tabular*}
\endgroup
\newpage
\subsubsection{端コンボ}\begingroup
 \renewcommand{\arraystretch}{1.2}
\begin{tabular*}{15.1cm}{@{\extracolsep{\fill}}|p{3em}||p{12.9cm}|}\hline
\multicolumn{2}{|p{14.6cm}|}{
\PG{0}\ \ \ \DG{0}
}\\\bhline{2pt}
コンボ&\command{近\C\ or\ 近\D}\ \Cancel\ \command{\tatsu\ +\ \B}\\
& \migi\ \command{\vtame\ +\ \D}\ \Cancel\ \command{$\downarrow$\ +\ \D}
%%コンボルート
\\\hline
補足&
端ならゲージを使わずに\command{\vtame\ +\ \D}がつながる。強制ダウン。
%%補足
\\\hline\hline
\multicolumn{2}{|p{14.6cm}|}{
\PG{1}\ \ \ \DG{0}
}\\\bhline{2pt}
コンボ&\command{近\C\ or\ 近\D}\ \Cancel\ \command{\tatsu\ +\ \B}\ \migi\
\command{\vtame\ +\ \B\D}\
%%コンボルート
\\\hline
補足&
倒しきれる時用。強制ダウン。
%%補足
\\\hline\hline
\multicolumn{2}{|p{14.6cm}|}{
\PG{2}\ \ \ \DG{0}
}\\\bhline{2pt}
コンボ&\command{$\downarrow$ \B}$\times\ 1\sim\ 3$\ \Cancel\ \command{立\B}\ \Cancel\
\command{\tatsu\ +\ \B\D}\\
& \migi\ \command{$\rightarrow\ \rightarrow$\ +\ \A$}\
\Cancel\ \command{\tatsu\ +\ \B\D}\\
& \migi\ \command{\vtame\ +\ \D}\ \Cancel\
\command{$\downarrow$\ +\ \D}
%%コンボルート
\\\hline
補足&
中央コンボからも端に運べた場合はコレ。強制ダウン。
%%補足
\\\hline\hline
\multicolumn{2}{|p{14.6cm}|}{
\PG{2}\ \ \ \DG{50}
}\\\bhline{2pt}
コンボ&\command{$\downarrow$ \B}$\times\ 1\sim\ 3$\ \Cancel\ \command{立\B}\ \Cancel\
\command{\tatsu\ +\ \B\D}\\
& \migi\ \command{$\rightarrow\ \rightarrow$\ +\ \A$}\
\Cancel\ \command{\tatsu\ +\ \B\D}\ \migi\ \command{\vtame\ +\ \B}\\
& \DC\ \command{\hado\ +\ \B\ or\ \D}\ \migi\
\command{\tatsu\ +\ \B}\\
& \migi\ \command{\vtame\ +\ \D}\ \Cancel\ 
\command{$\downarrow$\ +\ \D}
%%コンボルート
\\\hline
補足&
\command{\hado\ +\ \B\ or\ \D}は空振って着地する。そこまでダメージは伸びない。強制ダウン。
%%補足
\\\bhline{2pt}
\end{tabular*}
\endgroup
\newpage
\subsection{ホア}
\subsubsection{中央コンボ}
\begingroup
 \renewcommand{\arraystretch}{1.2}
\begin{tabular*}{15.1cm}{@{\extracolsep{\fill}}|p{3em}||p{12.9cm}|}\hline
\multicolumn{2}{|p{14.6cm}|}{
\PG{0}\ \ \ \DG{0}
}\\\bhline{2pt}
コンボ&
\command{近\C\ or\ \D}\ \Cancel\ \command{$\searrow$\ \B}\
\Cancel \command{\syoryu\ +\ \D}%%コンボルート
\\\hline
補足&
基本コンボ
\command{$\searrow$\ +\ \B}からのキャンセルは一瞬遅れてもOK%%補足
\\\bhline{2pt}%%\hline\hline
コンボ&
(\command{$\downarrow$ \B\ or\ 立\A}\ $\times\ 1\ \sim\ 2$\Cancel\ )
\command{立 \A\ or\ 立\B}\ \Cancel \command{$\searrow$\ \B}\\
& \Cancel\
\command{\syoryu\ +\ \D}%%コンボルート
\\\hline
補足&
小技始動。
\command{$\searrow$\ +\ \B}からのキャンセルは一瞬遅れてもOK。以下すべて始動は小技でもOK%%補足
\\\hline\hline
\multicolumn{2}{|p{14.6cm}|}{
\PG{1}\ \ \ \DG{0}
}\\\bhline{2pt}
コンボ&
\command{近\C\ or\ \D}\ \Cancel\ \command{$\searrow$\ \B}\
\Cancel \command{\ryuko\ +\ \A\ or\ \C}%%コンボルート
\\\hline
補足&
1ゲージのみ使用した場合の最大。\command{\ryuko\ +\ \C}だとキャンセルが遅いとつながらない。\\
&\command{\ryuko\ +\ \A}の場合は早すぎると届かないためヒットを確認して遅めにキャンセルする。\\\bhline{2pt}%%補足
コンボ&
\command{近\C\ or\ \D}\ \Cancel\ \command{$\searrow$\ \B}\
\Cancel \command{\hado\ \hado \ +\ \B\ or\ \D}%%コンボルート
\\\hline
補足&
早めにキャンセルしても遅めにキャンセルしても基本つながる\\\hline\hline
\multicolumn{2}{|p{14.6cm}|}{
\PG{1}\ \ \ \DG{50}
}\\\bhline{2pt}
コンボ&
\command{近\C\ or\ \D}\ \Cancel\ \command{$\searrow$\ \B}\
\Cancel \command{\syoryu \ +\ \D}\ \DC\ \command{\tatsu \ +\ \B}\\
&\migi\ \command{\hado\ \hado\ +\ \B\ or\ \D} %%コンボルート
\\\hline
補足&
簡単でそこそこ減る。ダメージ・有利時間重視\\\bhline{2pt}
コンボ&
\command{近\C\ or\ \D}\ \Cancel\ \command{$\searrow$\ \B}\
\Cancel \command{\syoryu \ +\ \D}\ \DC\ \command{\tatsu \ +\ \B}\\
&\migi\ \command{近\D}\ \Cancel \command{\syoryu\ +\ \B\D}\ \Cancel\
\command{$\rightarrow$\ \B\ or\ D}\ $\times 2$%%コンボルート
\\\hline
補足&
運び重視%%補足
\\\hline\hline
\multicolumn{2}{|p{14.6cm}|}{
\PG{2}\ \ \ \DG{0}
}\\\bhline{2pt}
コンボ&
\command{近\C\ or\ \D}\ \Cancel\ \command{$\searrow$\ \B}\
\Cancel \command{\hado\ \hado\ +\ \B\D} %%コンボルート
\\\hline
補足&
早めにキャンセルしても遅めにキャンセルしてもつながる。\\\hline\hline
\multicolumn{2}{|p{14.6cm}|}{
\PG{2}\ \ \ \DG{50}
}\\\bhline{2pt}
コンボ&
\command{近\C\ or\ \D}\ \Cancel\ \command{$\searrow$\ \B}\ \Cancel
\command{\syoryu \ +\ \D}\ \DC\ \command{\tatsu \ +\ \B}\\
&\migi\ \command{\hado\ \hado\ +\ \B\D} %%コンボルート
\\\hline
補足&
ほぼ端から逆の端まで運んで安全にドリンクが飲める。が、ゲージ消費過多\\\hline\hline
\end{tabular*}
\endgroup
\newpage
\subsubsection{端コンボ}
\begingroup
 \renewcommand{\arraystretch}{1.2}
\begin{tabular*}{15.1cm}{@{\extracolsep{\fill}}|p{3em}||p{12.9cm}|}\hline
\multicolumn{2}{|p{14.6cm}|}{
\PG{1}\ \ \ \DG{0}
}\\\bhline{2pt}
コンボ&
\command{近\C\ or\ \D}\ \Cancel\ \command{$\searrow$\ \B}\ \Cancel\
\command{\hado\ +\ \A\C}\\
&\migi\ \command{$\rightarrow$\ \hado\ $\nearrow$\ +\B}\ \migi\
\command{\syoryu\ +\ \D}
%%コンボルート
\\\hline
補足&
\command{$\rightarrow$\ \hado\ $\nearrow$\ +\B}は、\command{\hado\ +\
\A\C}後コマンドを入れて\\
&$\nearrow$に入れたまま\B を押しっぱなしにすればいい。ゲージをそこそこ回収する。
%%補足
\\\hline\hline
\multicolumn{2}{|p{14.6cm}|}{
\PG{1}\ \ \ \DG{50}
}\\\bhline{2pt}
コンボ&
\command{近\C\ or\ \D}\ \Cancel\ \command{$\searrow$\ \B}\ \Cancel\
\command{\hado\ +\ \A\C}\\
&\migi\ \command{$\rightarrow$\ \hado\ $\nearrow$\ +\B}\ \migi\
\command{\syoryu\ +\ \D}\ \DC\ \command{\tatsu\ +\ \D}\\
&\migi\ \command{\tatsu\ +\ \D}\ \migi\ \command{\syoryu\ +\ \D}\\ 
& \migi\ \command{近\C}
%%コンボルート
\\\hline
補足&コンボ中に1ゲージ回収する。
%%補足
\\\hline\hline
\multicolumn{2}{|p{14.6cm}|}{
\PG{2}\ \ \ \DG{0}
}\\\bhline{2pt}
コンボ&\command{近\C\ or\ \D}\ \Cancel\ \command{$\searrow$\ \B}\ \Cancel\
\command{\hado\ +\ \A\C}\ \migi\ \command{\tatsu\ +\ \B}\\
&\migi\ \command{\hado\ +\ \A}\ \migi\ \command{\hado\hado\ +\ \B\ or\ \D}
%%コンボルート
\\\hline
補足&
特になし
%%補足
\\\hline\hline
\multicolumn{2}{|p{14.6cm}|}{
\PG{2}\ \ \ \DG{50}
}\\\bhline{2pt}
コンボ&\command{近\C\ or\ \D}\ \Cancel\ \command{$\searrow$\ \B}\ \Cancel\
\command{\hado\ +\ \A\C}\\
&\migi\ \command{$\rightarrow$\ \hado\ $\nearrow$\ +\B}\ \migi\
\command{\syoryu\ +\ \D}\ \DC\ \command{\tatsu\ +\ \D}\\
&\migi\ \command{\tatsu\ +\ \D}\ \migi\ \command{\syoryu\ +\ \D}\\ 
& \migi\ \command{\hado\hado\ +\ \B\ or\ \D}\\
& (or\ \command{近\C}\ \Cancel\ \command{\tatsu\tatsu\ +\ \A\ or\ \C})
%%コンボルート
\\\hline
補足&
\command{近\C}\ \Cancel\ \command{\tatsu\tatsu\ +\ \A\ or\
\C}の場合、ドリンクを飲んでも微有利から五分。コンボ中に1ゲージ回収。
%%補足
\\\bhline{2pt}%%\hline\hline
\end{tabular*}
\endgroup
\newpage
\subsection{ライデン}
\newpage
\part{K'チーム}%%チーム名
\section{コマンド}
\subsection{K'}
\begin{itembox}[l]{特殊技}
\begin{tabular}{lll}
ワンインチ&$\rightarrow$\ +\ \A&%%備考%%
\end{tabular}
\end{itembox}
\begin{itembox}[l]{必殺技}
\begin{tabular}{lll}
アイントリガー&\hado\ +\ \A\ or\ \C&弾判定/追加入力で派生\\%%備考%%
セカンドシュート&(アイントリガー中)$\rightarrow$\ +\ \B&EX版弾貫通\\%%備考%%
セカンドシェル&(アイントリガー中)$\rightarrow$\ +\ \D&\\%%備考%%
ブラックアウト&(アイントリガー中)$\leftarrow$\ +\ \B\ or\ \D&\\%%備考%%
ブラックアウト&\hado\ +\ \B\ or\ \D&EX版打撃無敵\\%%備考%%
クロウバイツ&\syoryu\ +\ \A\ or\ \D&弱上半身無敵/強全身無敵\\%%備考%%
&&強$\rightarrow$\ +\ \B\ or\ \D で追加\\
ミニッツスパイク&\tatsu\ +\ \B\ or\ \D&空中可\\
&&空中版なんでも判定\\%%備考%%
ナロウスパイク&(地上MS)\tatsu\ +\ \B\ or\ \D&下段%%備考%%
\end{tabular}
\end{itembox}
※MS…ミニッツスパイク\\
\begin{itembox}[l]{超必殺技}
\begin{tabular}{lll}
ヒートドライブ&\hado\ \hado\ +\ \A\ or\ \C&EX版無し/タメ可\\%%備考%%
チェーンドライブ&\ryuko\ +\ \A\ or\ \C&サングラスHit時のみ演出\\%%備考%%
&&EX版なんでも判定
\end{tabular}
\end{itembox}
\begin{itembox}[l]{NEOMAX超必殺技}
\begin{tabular}{lll}
ハイパーチェーンドライブ&\gyakuyoga\ $\times\ 2$\ +\ $\A \C$&%%備考%%
\end{tabular}
\end{itembox}
\newpage
\subsection{クーラ}
\begin{itembox}[l]{特殊技}
\begin{tabular}{lll}
ワンインチ&$\rightarrow$\ +\ \A&\\%%備考%%
スライダーシュート&$\searrow$\ +\ \B&下段%%備考%%
\end{tabular}
\end{itembox}
\begin{itembox}[l]{必殺技}
\begin{tabular}{lll}
レイ・スピン&\tatsu\ +\ \B\ or\ \D&追加入力で派生\\%%備考%%
スタンド&(レイ・スピン中)$\rightarrow$\ +\ \B&弾判定\\%%備考%%
シット&(レイ・スピン中)$\rightarrow$\ +\ \D&下段\\%%備考%%
クロウバイツ&\syoryu\ +\ \A\ or\ \C&弱完全無敵\\%%備考%%
ダイアモンドブレス&\hado\ +\ \A\ or\ \C&弾判定\\%%備考%%
カウンターシェル&\tatsu\ +\ \A\ or\ \C&弾反射+弾貫通%%備考%%
\end{tabular}
\end{itembox}
\begin{itembox}[l]{超必殺技}
\begin{tabular}{lll}
ダイアモンドエッジ&\hado\ \had\ +\ \A\ or\ \C&\\%%備考%%
フリーズエクスキュージョン&\gyakuyoga\ $\times\ 2$\ +\ \A\ or\ \C&2PG消費\\
&&弾貫通%%備考%%
\end{tabular}
\end{itembox}
\begin{itembox}[l]{NEOMAX超必殺技}
\begin{tabular}{lll}
ネオフリーズエクスキュージョン&\gyakuyoga\ $\times\ 2$\ +\ $\B\ \D$&弾貫通%%備考%%
\end{tabular}
\end{itembox}
\newpage
\subsection{マキシマ}
\begin{itembox}[l]{特殊技}
\begin{tabular}{lll}
$M$9型\ マキシマ・ミサイル(試作)&$\searrow$\ +\ \C&%%備考%%
\end{tabular}
\end{itembox}
\begin{itembox}[l]{必殺技}
\begin{tabular}{lll}
$M$4型 ベイパーキャノン&\tatsu\ +\ \A\ or\ \C&強版上半身GP\\%%備考%%
&&EX版壁バウンド\\
$M$4型 ベイパーキャノン(空中)&(空中で)\tatsu\ +\ \A\ or\ \C&EX版全身GP\\%%備考%%
$M$19型 ブリッツキャノン&\syoryu\ +\ \B\ or\ \D&EX版全身GP\\%%備考%%
マキシマプレス&\gyakuyoga\ +\ \B\ or\ \D&弱1F投げ\\%%備考%%
&&強移動投げ\\
&&EX版打撃技\\
追加攻撃&(MP中)\tatsu\ +\ \A\ or\ \C&
\end{tabular}
\end{itembox}
※GP…ガードポイント\\
※MP…マキシマプレス\\
\begin{itembox}[l]{超必殺技}
\begin{tabular}{lll}
ダブルベイパーキャノン&\hado\ \hado\ +\ \A\ or\ \C&通常版全身GP\\%%備考%%
&&EX版全身無敵
\end{tabular}
\end{itembox}
\begin{itembox}[l]{NEOMAX超必殺技}
\begin{tabular}{lll}
マキシマレーザー&\gyakuyoga\ $\times\ 2$\ +\ \A\C&最終段のみガーキャン可%%備考%%
\end{tabular}
\end{itembox}
\newpage
\section{コンボ}
\subsection{K'}
\subsection{クーラ}
\newpage
\subsection{マキシマ}
\subsubsection{中央・端コンボ}
\begingroup
 \renewcommand{\arraystretch}{1.2}
\begin{tabular*}{15.1cm}{@{\extracolsep{\fill}}|p{3em}||p{12.9cm}|}\hline
\multicolumn{2}{|p{14.6cm}|}{
\PG{0}\ \ \ \DG{0}
}\\\bhline{2pt}
コンボ&
\command{近\C\ or\ 近\D}\ \Cancel\ \command{$\searrow$\ \C}\ \Cancel\
\command{\tatsu\ +\ \A}
%%コンボルート
\\\hline
補足&
ヒット確認も容易で、出しきってガードされてもローリスク。
%%補足
\\\bhline{2pt}%%\hline\hline
コンボ&
\command{$\downarrow$\ \B}\ \Cancel\
\command{$\downarrow$\A}\ \Cancel \command{$\searrow$\ \C}\ \Cancel\
\command{\tatsu\ +\ \A}
%%コンボルート
\\\hline
補足&
小技始動。遠いと届かない。\command{$\searrow$\ \C}につなぐなら以下のコンボはこの始動でも可。
%%補足
\\\hline\hline
\multicolumn{2}{|p{14.6cm}|}{
\PG{1}\ \ \ \DG{0}
}\\\bhline{2pt}
コンボ&
\command{近\C\ or\ 近\D}\ \Cancel\ \command{$\searrow$\ \C}\ \Cancel\
\command{\hado\ \hado\ +\ \A\ or\ \C}
%%コンボルート
\\\hline
補足&
もう少し高いのがあるが、これが一番簡単。\command{$\searrow$\ \C}無しでも小技からつながる。
%%補足
\\\bhline{2pt}%%\hline\hline
コンボ&
\command{近\C\ or\ 近\D}\ \Cancel\ \command{$\searrow$\ \C}\ \Cancel\
\command{\tatsu\ +\ \A\C}\ \migi \command{$\searrow$\ \C}\ \\
&\Cancel\ \command{\tatsu\ +\ \C}
%%コンボルート
\\\hline
補足&
\command{\tatsu\ +\ \A\C}が壁バウンドを誘発するので追撃可能。ただし、位置次第でダッシュが必要だったり位置が入れ替わったりする。\\
&\command{$\searrow$\ \C}無しでも小技からつながる。
%%補足
\\\hline\hline
\multicolumn{2}{|p{14.6cm}|}{
\PG{1}\ \ \ \DG{50}
}\\\bhline{2pt}
コンボ&
\command{近\C\ or\ 近\D}\ \Cancel\ \command{$\searrow$\ \C}\ \Cancel\
\command{\tatsu\ +\ \A}\ \\
&\SC\ \command{\hado\ \hado\ +\ \A\ or\ \C}
%%コンボルート
\\\hline
補足&
基本コンボを伸ばした形。やや忙しい。
%%補足
\\\bhline{2pt}%%\hline\hline
コンボ&
\command{\gyakuyoga\ +\ \B\ or\ \D} \Cancel\
\command{\tatsu\ +\ \A}\ \\
&\SC\ \command{\hado\ \hado\ +\ \A\ or\ \C}
%%コンボルート
\\\hline
補足&
コマ投げから。位置次第で結構忙しい。
%%補足
\\\hline\hline
\multicolumn{2}{|p{14.6cm}|}{
\PG{2}\ \ \ \DG{0}
}\\\bhline{2pt}
コンボ&
\command{近\C\ or\ 近\D}\ \Cancel\ \command{$\searrow$\ \C}\ \Cancel\
\command{\hado\ \hado\ +\ \A\C}
%%コンボルート
\\\hline
補足&
簡単で減る。\command{$\searrow$\ \C}無しでも小技からつながる。
%%補足
\\\hline\hline
\multicolumn{2}{|p{14.6cm}|}{
\PG{2}\ \ \ \DG{50}
}\\\bhline{2pt}
コンボ&
\command{近\C\ or\ 近\D}\ \Cancel\ \command{$\searrow$\ \C}\ \Cancel\
\command{\tatsu\ +\ \A}\ \\
&\SC\ \command{\hado\ \hado\ +\ \A\C}
%%コンボルート
\\\hline
補足&
基本コンボを伸ばした形。やや忙しい。
%%補足
\\\bhline{2pt}%%\hline\hline
コンボ&
\command{\gyakuyoga\ +\ \B\ or\ \D} \Cancel\
\command{\tatsu\ +\ \A}\ \\
&\SC\ \command{\hado\ \hado\ +\ \A\C}
%%コンボルート
\\\hline
補足&
コマ投げから。位置次第で結構忙しい。
%%補足
\\\bhline{2pt}
\end{tabular*}
\endgroup
\newpage
\newpage
\part{エリザベートチーム}%%チーム名
\section{コマンド}
\subsection{エリザベート}
\begin{itembox}[l]{特殊技}
\begin{tabular}{lll}
アン・ク・ドゥ・ピエ&$\rightarrow$\ +\ \B&%%備考%%
\end{tabular}
\end{itembox}
\begin{itembox}[l]{必殺技}
\begin{tabular}{lll}
エタンセル&\hado\ +\ \A\ or\ \C&弾判定\\%%備考%%
レヴェリー・スエテ&\hado\ +\ \B\ or\ \D&移動技/EX版打撃無敵\\%%備考%%
&&全強度動作中キャンセル可\\
レヴェリー・プリエ&\tatsu\ +\ \B\ or\ \D&移動技/EX版打撃無敵\\%%備考%%
&&全強度動作中キャンセル可\\
ミストラル&\tenti\ +\ \A\ or\ \C&投げ技/EX版発生まで無敵\\%%備考%%
クー・ド・ヴァン&\syoryu\ +\ \A\ or\ \C&%%備考%%
\end{tabular}
\end{itembox}
\begin{itembox}[l]{超必殺技}
\begin{tabular}{lll}
ノーブル・ブラン&\hado\ \hado\ +\ \A\ or\ \C&弾貫通\\%%備考%%
グラン・ラファール&\ryuko\ +\ \A\ or\ \C&なんでも判定/EX版無し%%備考%%
\end{tabular}
\end{itembox}
\begin{itembox}[l]{NEOMAX超必殺技}
\begin{tabular}{lll}
エトワール・フィラント&\orochi\ +\ $\B\ \D$&当て身技\\%%備考%%
&&MAXCancel時打撃技
\end{tabular}
\end{itembox}
\newpage
\subsection{デュオロン}
\begin{itembox}[l]{必殺技}
\begin{tabular}{lll}
幻夢拳&$\rightarrow$\ +\ \A&空中可/空中EX版なんでも判定\\%%備考%%
&&地上版幻夢脚以外でキャンセル可\\
幻夢脚&$\rightarrow$\ +\ \B&空中可/空中版中段\\%%備考%%
&&地上版相手をサーチ\\
&&地上版超必殺技以上でキャンセル可\\
飛毛脚・前&\hado\ +\ \B\ or\ \D&EX版打撃無敵\\%%備考%%
飛毛脚・後&\hado +\ \B\ or\ \D&EX版打撃無敵\\%%備考%%
&&動作中幻夢脚でキャンセル可\\
捨己従竜&\hado\ +\ \A\ or\ \C&3回まで連続入力可\\%%備考%%
&&3段目飛毛脚・前でキャンセル可\\
&&EX版一部必殺技でキャンセル可\\
呪怨死魂&\tatsu\ +\ \A\ or\ \C&EX版弾貫通%%備考%%
\end{tabular}
\end{itembox}
\begin{itembox}[l]{超必殺技}
\begin{tabular}{lll}
秘伝・幻夢爆吐死魂&\orochi\ +\ \A\ or\ \C&%%備考%%
\end{tabular}
\end{itembox}
\begin{itembox}[l]{NEOMAX超必殺技}
\begin{tabular}{lll}
秘伝・幻夢呪怨死魂葬&\gyakuyoga\ $\times\ 2$\ +\ \A\C&1F投げ\\%%備考%%
&&MAXキャンセル時\\
&&打撃技に変化
\end{tabular}
\end{itembox}
\newpage
\subsection{シェン}
\begin{itembox}[l]{特殊技}
\begin{tabular}{lll}
斧旋脚&$\rightarrow$\ +\ \B&%%備考%%
\end{tabular}
\end{itembox}
\begin{itembox}[l]{必殺技}
\begin{tabular}{lll}
激拳&\hado\ +\ \A\ or\ \C&強タメ可\\
&&タメ中\B\ or\ \Dでキャンセル可\\%%備考%%
&&EX版無敵\\
伏虎撃&\tatsu\ +\ \A&\A\C でEX版が出る/中段\\%%備考%%
降龍撃&(伏虎撃中)\hado\ +\ \A&\\%%備考%%
転連拳&\hado\ +\ \B\ or\ \D&\\%%備考%%
弾拳&\tenti\ +\ \A\ or\ \C&投げ技/EX版発生まで無敵\\%%備考%%
弾拳(弾き)&\tatsu\ +\ \C&EX版無し/弾消し判定のみ\\%%備考%%
&&弾消し時キャンセル可
\end{tabular}
\end{itembox}
\begin{itembox}[l]{超必殺技}
\begin{tabular}{lll}
絶・激拳&\hado\ \hado\ +\ \A\ or\ \C&スライドダウン\\%%備考%%
爆真&$\C\ \cdot\ \A\ \cdot \B\ \cdot\ \C$&全身無敵/一時強化%%備考%%
\end{tabular}
\end{itembox}
\begin{itembox}[l]{NEOMAX超必殺技}
\begin{tabular}{lll}
天将爆真激&\ryuko\ +\ \A\C&ヒット時のみ演出%%備考%%
\end{tabular}
\end{itembox}
\newpage
\section{コンボ}
\subsection{エリザベート}
\subsection{デュオロン}
\subsection{シェン}
\newpage
\part{エディットキャラクター}%%チーム名
\section{コマンド}
\subsection{ビリー}
\begin{itembox}[l]{特殊技}
\begin{tabular}{lll}
大回転蹴り&$\rightarrow$\ +\ \A&\\%%備考%%
竿打ち&$\leftarrow$\ +\ \A&\\%%備考%%
棒高跳び蹴り&$\rightarrow$\ +\ \B&しゃがみに当たらない%%備考%%
\end{tabular}
\end{itembox}
\begin{itembox}[l]{必殺技}
\begin{tabular}{lll}
三節棍中段打ち&\yoga\ +\ \A\ or\ \C&\\%%備考%%
火炎三節棍中段打ち&(強/EX三節棍中段打ち中)&\\
&\hado\ +\ \A\ or\ \C&\\
旋風棍&\A\ or\ \C\ 連打&出掛かりをキャンセル可能\\%%備考%%
雀落とし&\tatsu\ +\ \A\ or\ \C&強ヒット時キャンセル可\\%%備考%%
強襲飛翔棍&\syoryu\ +\ \B\ or\ \D&EX版無敵\\&&EX版下降部分相手サーチ%%備考%
\end{tabular}
\end{itembox}
\begin{itembox}[l]{超必殺技}
\begin{tabular}{lll}
超火炎旋風棍&\ryuko\ +\ \A\ or\ \C&弾貫通%%備考%%
\end{tabular}
\end{itembox}
\begin{itembox}[l]{NEOMAX超必殺技}
\begin{tabular}{lll}
大紅蓮螺旋棍&\hado\ \hado\ +\ \A\C&弾貫通%%備考%%
\end{tabular}
\end{itembox}
\newpage
\subsection{アッシュ}
\begin{itembox}[l]{特殊技}
\begin{tabular}{lll}
フロレアール&$\leftarrow$\ +\ \B&下段無敵\\%%備考%%
フロレアール(後方)&$\leftarrow$\ +\ \D&下段無敵%%備考%%
\end{tabular}
\end{itembox}
\begin{itembox}[l]{必殺技}
\begin{tabular}{lll}
ヴァントーズ&\htame\ +\ \A\ or\ \C&EX版弾貫通\\%%備考%%
ジェルミナール カプリス&\htame\ +\ \B\ or\ \D&\\%%備考%%
ニヴォース&\vtame\ +\ \B\ or\ \D&強全身無敵\\%%備考%%
ジェニー&\tatsu\ +\ \A\ or\ \B\ &設置技\\%%備考%%
&\ \ \ \ \ \ \ \ \ \ or\ \C\ or\ \D &相手が近づくと爆発\\
&&EX \A\C 版目の前に設置\\
&&EX \B\D 版相手をサーチ
\end{tabular}
\end{itembox}
\begin{itembox}[l]{超必殺技}
\begin{tabular}{lll}
テルミドール&\hado\ \hado\ +\ \A\ or\ \C&EX版無し/弾貫通\\%%備考%%
ブリュヴィオーズ&\hado\ \hado\ +\ \B\ or\ \D&\\%%備考%%
サン・キュロット&$\A\ \cdot\ \B\ \cdot\ \C\ \cdot\ \D$&全身無敵/一時強化\\%%備考%%
ジェルミナール&(強化中)\orochi\ +\ \A\C&消費ゲージ無し\\
&&必殺技以上一時封印%%備考%%
\end{tabular}
\end{itembox}
\begin{itembox}[l]{NEOMAX超必殺技}
\begin{tabular}{lll}
フリュティドール&\gyakuyoga\ $\times\ 2$\ +\ $\B\ \D$&1F投げ\\
&&MAXキャンセル時打撃技%%備考%%
\end{tabular}
\end{itembox}
\newpage
\subsection{サイキ}
\begin{itembox}[l]{特殊技}
\begin{tabular}{lll}
陰&(空中で)$\rightarrow$\ +\ \B&空中ヒット後追撃不可\\%%備考%%
發&(空中で)$\leftarrow$\ +\ \B&
\end{tabular}
\end{itembox}
\begin{itembox}[l]{必殺技}
\begin{tabular}{lll}
鬼抑ノ月&\hado\ +\ \A\ or\ \C&\\%%備考%%
刎釣瓶ノ鉈&\syoryu\ +\ \B\ or\ \D&弱上半身無敵/強全身無敵\\%%備考%%
笠研ノ槌&(強/EX刎釣瓶ノ鉈ヒット時)&\\
&\A\ or\ \B\ or\ \C\ or\ \D押しっぱなし&\\%%備考%%
臂折ノ楔&\tatsu\ +\ \B\ or\ \D&\\%%備考%%
裏七里&(強/EX臂折ノ楔ヒット時)&移動技\\%%備考%%
&\A\ or\ \B\ or\ \C\ or\ \D押しっぱなし&\\%%備考%%
七里駆&$\downarrow\ \downarrow$\ +\A\ or\ \B\ or\ \C\ or\ \D&%移動技%備考%%
\end{tabular}
\end{itembox}
\begin{itembox}[l]{超必殺技}
\begin{tabular}{lll}
鷲羽落&\hado\ \hado\ +\ \B\ or\ \D&EX版無し/ヒット時演出\\%%備考%%
闇落&(空中で)\hado \hado\ +\ \B\ or\ \D&EX版無し/中段/ヒット時演出\\%%備考%%
常闇ノ船&\hado\ \hado\ +\ \A\ or\ \C&EX版無し/弾貫通/補正無視\\%%備考%%
去龍ノ澱&\gyakuyoga\ $\times\ 2$\ +\ \A\ or\ \C&1F投げ/体力回復\\%%備考%%
\end{tabular}
\end{itembox}
\begin{itembox}[l]{NEOMAX超必殺技}
\begin{tabular}{lll}
神集&\orochi\ +\ \A\C&全画面%%備考%%
\end{tabular}
\end{itembox}
\newpage
\section{コンボ}
\subsection{ビリー}
\newpage
\subsection{アッシュ}
\subsubsection{中央コンボ}
\begingroup
 \renewcommand{\arraystretch}{1.2}
\begin{tabular*}{15.1cm}{@{\extracolsep{\fill}}|p{3em}||p{12.9cm}|}\hline
\multicolumn{2}{|p{14.6cm}|}{
\PG{0}\ \ \ \DG{0}
}\\\bhline{2pt}
コンボ&
\command{近\C}\ \Cancel\ \command{$\leftarrow$\ \B}\ \migi\ \command{\vtame\ +\
\B}
%%コンボルート
\\\hline
補足&
基本コンボ。距離が遠いとつながらない。\\
&\command{$\leftarrow$\ \B}\ \migi\ \command{\vtame\ +\
\B}は\command{$\leftarrow$\ \B}を入力してすぐ$\downarrow$\
方向にタメ始め、先行入力を利用してつなげる。
%%補足
\\\bhline{2pt}%%\hline\hline
コンボ&
\command{$\downarrow$\ \B}\ $\times\ 1\ \sim\ 2$ \Cancel\ \command{$\leftarrow$\
\B}\ \migi\
\command{\vtame\ +\ \B}
%%コンボルート
\\\hline
補足&
小技始動基本コンボ。距離が遠いとつながらない。\command{$\leftarrow$\
\B}を省いてもいい。以後\command{$\leftarrow$\
\B}につなぐならばこの始動も可。
%%補足
\\\hline\hline
\multicolumn{2}{|p{14.6cm}|}{
\PG{1}\ \ \ \DG{0}
}\\\bhline{2pt}
コンボ&
\command{近\C}\ \Cancel\ \command{$\leftarrow$\ \B}\ \migi\ \command{\vtame\ +\
\B\D}
%%コンボルート
\\\hline
補足&
特になし
%%補足
\\\hline\hline
\multicolumn{2}{|p{14.6cm}|}{
\PG{2}\ \ \ \DG{0}
}\\\bhline{2pt}
コンボ&
\command{近\C}\ \Cancel\ \command{$\leftarrow$\ \B}\ \migi\ \command{\hado\
\hado\ +\ \B\D}
%%コンボルート
\\\hline
補足&
タメがいらないので簡単。そこそこの減り。アッシュに慣れない内はコレで安定。
%%補足
\\\bhline{2pt}%%\hline\hline
\end{tabular*}
\endgroup
\subsubsection{端コンボ}
\begingroup
 \renewcommand{\arraystretch}{1.2}
\begin{tabular*}{15.1cm}{@{\extracolsep{\fill}}|p{3em}||p{12.9cm}|}\hline
\multicolumn{2}{|p{14.6cm}|}{
\PG{1}\ \ \ \DG{0}
}\\\bhline{2pt}
コンボ&
\command{近\C}\ \Cancel\ \command{\tatsu\ +\ \A\C}\ \migi\ \command{\htame\ +\
\A}\ $\times\ 3$\\
& \migi\ \command{\vtame\ +\ \D}
%%コンボルート
\\\hline
補足&
すべて最速。先行入力を使うと楽。1ゲージのみのコンボとしては破格。
%%補足
\\\bhline{2pt}%%\hline\hline
\end{tabular*}
\endgroup
\newpage
\subsection{サイキ}
\newpage
\part{裏キャラクター}%%チーム名
\section{コマンド}
\subsection{炎庵}
\begin{itembox}[l]{特殊技}
\begin{tabular}{lll}
外式・夢弾&$\rightarrow$\ +\ \A&\A\ 追加で二段目発生\\
外式・轟斧 陰"死神"&$\rightarrow$\ +\ \B&中段\\%%備考%%
外式・百合折り&(空中で)$\leftarrow$\ +\ \B&バクステ中可%%備考%%
\end{tabular}
\end{itembox}
\begin{itembox}[l]{必殺技}
\begin{tabular}{lll}
百式・鬼焼き&\syoryu\ +\ \A\ or\ \C&弱上半身無敵/強全身無敵\\%%備考%%
百八式・闇払い&\hado\ +\ \A\ or\ \C&EX版弾貫通/Hit時拘束\\%%備考%%
百弐拾七式・葵花&\tatsu\ +\ \A\ or\ \C&3回まで追加入力可\\%%備考%%
弐百拾弐式・琴月 陰&\gyakuyoga\ +\ \B\ or\ \D&EX版ダウン追い打ち\\%%備考%%
屑風&\tenti\ +\ \A\ or\ \C&投げ技/EX版無敵%%備考%%
\end{tabular}
\end{itembox}
\begin{itembox}[l]{超必殺技}
\begin{tabular}{lll}
禁千弐百十壱式・八稚女&\ryuko\ +\ \A\ or\ \C&EX版スライドダウン\\%%備考%%
裏参百壱拾六式・豺華&(八稚女)(\ \hado\ )\times 4\ +\ \A\C&補正無視\\
裏千弐百七式・闇削ぎ&\hado\ \hado\ +\ \A\ or\ \C&EX版無し\\
&&初段ヒット時ロック%%備考%%
\end{tabular}
\end{itembox}
\begin{itembox}[l]{NEOMAX超必殺技}
\begin{tabular}{lll}
裏千弐拾九式・焔甌&\orochi +\ \A\C&投げ技%%備考%%
\end{tabular}
\end{itembox}
\newpage
\subsection{ネスツ京}
\begin{itembox}[l]{特殊技}
\begin{tabular}{lll}
外式・轟斧 陽&$\rightarrow$\ +\ \B&中段\\%%備考%%
八拾八式&$\searrow$\ +\ \D&下段/二段技\\%%備考%%
外式・奈落落とし&(空中で)$\swarrow$\ or\ $\downarrow$\ or\ $\searrow$\ +\
\C&中段/バクステ中可%%備考%%
\end{tabular}
\end{itembox}
\begin{itembox}[l]{必殺技}
\begin{tabular}{lll}
百式・鬼焼き&\syoryu\ +\ \A\ or\ \C&弱上半身無敵/強全身無敵\\%%備考%%
&&全強度上半身GP\\
R.E.\D.Kick&\gyakusyoryu\ +\ \B\ or\ \D&EX版相手サーチ\\%%備考%%
弐百拾弐式・琴月 陽&\gyakuyoga\ +\ \A\ or\ \C&下段無敵\\%%備考%%
&&EX版下段無敵\\
&&EX版上半身GP\\
七拾五式・改&\hado\ +\ \B\ or\ \D&\B\ or\ \D 追加で二段目発生\\%%備考%%
百拾四式・荒咬み&\hado\ +\ \A&上半身GP\\%%備考%%
&(EX毒咬み中)\hado\ +\ \A&\\
百弐拾八式・九傷&(荒咬み中)\hado\ +\ \A&\\%%備考%%
百弐拾五式・七瀬&(九傷中)\B\ or\ \D&\\%%備考%%
百弐拾七式・八錆&(九傷中)\A\ or\ \C&中段\\%%備考%%
百弐拾七式・八錆&(荒咬み中)\gyakuyoga&中段\\
&\ +\ \A\ or\ \C&\\%%備考%%
百弐拾五式・七瀬&(八錆中)\B\ or\ \D&\\%%備考%%
外式・砌穿ち&(八錆中)$\downarrow$\ +\ \A\ or\ \C&ダウン追い打ち\\%%備考%%
百拾五式・毒咬み&\hado\ +\ \C&上半身GP\\%%備考%%
&&\A\C でEX毒咬み\\
四百壱式・罪詠み&(毒咬み中)&\\
&\gyakuyoga\ +\ \A\ or\ \C&\\%%備考%%
四百弐式・罰詠み&(罪詠み中)$\rightarrow$\ +\ \A\ or\ \C&\\%%備考%%
百式・鬼焼き&(罰詠み中)\syoryu\ +\ \A\ or\ \C&%%備考%%
\end{tabular}
\end{itembox}
\begin{itembox}[l]{超必殺技}
\begin{tabular}{lll}
裏百八式・大蛇薙&\orochi\ +\ \A\ or\ \C&弾貫通/上半身無敵\\%%備考%%
最終決戦奥義/無式&\hado\ \hado\ +\ \A\ or\ \C&EX版無し%%備考%%
\end{tabular}
\end{itembox}
\begin{itembox}[l]{NEOMAX超必殺技}
\begin{tabular}{lll}
最終決戦秘奥義/十拳&\hado\ \hado\ +\ $\B\ \D$&%%備考%%
\end{tabular}
\end{itembox}
\newpage
\subsection{Mr.カラテ}
\begin{itembox}[l]{特殊技}
\begin{tabular}{lll}
正拳三段突き&$\rightarrow$\ +\ \A&二段目までヒットバックなし\\%%備考%%
足刀蹴り&$\rightarrow$\ +\ \B&%%備考%%
\end{tabular}
\end{itembox}
\begin{itembox}[l]{必殺技}
\begin{tabular}{lll}
虎煌拳&\hado\ +\ \A\ or\ \C&EX版強制ガードクラッシュ\\%%備考%%
虎脚&(強虎煌拳中)$\rightarrow\ \rightarrow$&移動技\\%%備考%%
虎砲&\syoryu\ +\ \A\ or\ \C&弱全身無敵\\%%備考%%
ブレーキング&(強虎砲中)\A\B&\\%%備考%%
飛燕疾風脚&(空中で)\yoga\ +\ \B\ or\ \D&バクステ中可\\%%備考%%
暫烈拳&\zanretu +\ \A\ or\ \C&\\%%備考%%
翔乱脚&\gyakuyoga\ +\ \B\ or\ \D&移動投げ\\%%備考%%
覇極陣&\tatsu\ +\ \A\ or\ \C&当て身技%%備考%%
\end{tabular}
\end{itembox}
\begin{itembox}[l]{超必殺技}
\begin{tabular}{lll}
覇王至高権&$\rightarrow$\ \yoga\ +\ \A\ or\ \C&弾貫通/EX版なし\\%%備考%%
極限虎砲&\hado\ \hado\ +\ \B\ or\ \D&EX版なし\\%%備考%%
龍虎乱舞&\ryuko\ +\ \A\ or\ \C&%%備考%%
\end{tabular}
\end{itembox}
\begin{itembox}[l]{NEOMAX超必殺技}
\begin{tabular}{lll}
鬼神山峨撃&\orochi\ +\ \A\C&ヒット時演出%%備考%%
\end{tabular}
\end{itembox}
\newpage
\section{コンボ}
\subsection{炎庵}
\subsubsection{中央コンボ}
\begingroup
 \renewcommand{\arraystretch}{1.2}
\begin{tabular*}{15.1cm}{@{\extracolsep{\fill}}|p{3em}||p{12.9cm}|}\hline
\multicolumn{2}{|p{14.6cm}|}{
\PG{0}\ \ \ \DG{0}
}\\\bhline{2pt}
コンボ&
\command{近\C}\ \Cancel\ \command{$\rightarrow$\ \A}\ \Cancel\
\command{\tatsu\ +\ \C} $\times\ 3$\\\hline
補足&最初に覚えるべき基礎コンボ。\command{\tatsu\ +\ \A} $\times\
3$でもいいがこちらのほうが高い。\\\bhline{2pt}
コンボ&
\command{$\downarrow$\ \B}\ \Cancel\ \command{$\downarrow$\ \A}\ \Cancel\
\command{$\rightarrow$\ \A}\ \Cancel\ \command{\tatsu\ +\ \C} $\times\
3$\\\hline 補足&最初に覚えるべき小技始動基礎コンボ。以下すべてのコンボはこちらの始動でもOK\\\hline\hline
\multicolumn{2}{|p{14.6cm}|}{
\PG{0}\ \ \ \DG{50}
}\\\bhline{2pt}
コンボ&
\command{近\C}\ \Cancel\ \command{$\rightarrow$\ \A}\ \Cancel\
\command{\tatsu\ +\ \A\ or\ \C} $\times\ 2$\ \\
&\DC\
\command{\gyakuyoga\ +\ \B\ or\ \D}\\\hline
補足&ダメージはそこまで伸びないので倒しきれる時に\\\hline\hline 
\multicolumn{2}{|p{14.6cm}|}{
\PG{1}\ \ \ \DG{0}
}\\\hline
コンボ&
\command{近\C}\ \Cancel\ \command{$\rightarrow$\ \A}\ \Cancel\
\command{\ryuko\ +\A\ or\ \C}\\\hline
補足&ドライブゲージを使わない1ゲージコンボ。やや忙しい\\\hline\hline
\multicolumn{2}{|p{14.6cm}|}{
\PG{1}\ \ \ \DG{50}
}\\\hline
コンボ&
\command{近\C}\ \Cancel\ \command{$\rightarrow$\ \A}\ \Cancel\
\command{\tatsu\ +\ \A\ or\ \C} $\times\ 2$\ \\
&\SC\
\command{\ryuko\ +\ \A\ or\ \C}\\\hline
補足&ゲージが余ったらとりあえずこれ。\\\hline\hline
\multicolumn{2}{|p{14.6cm}|}{
\PG{2}\ \ \ \DG{0}
}\\\hline
コンボ&
\command{近\C}\ \Cancel\ \command{$\rightarrow$\ \A}\ \Cancel\
\command{\ryuko\ +\ \A\C}\\\hline
補足&以下4hit以上であれば\\
&\command{\ryuko\ +\ \A\ or\ \C}\ \Cancel \command{\hado\
$\times\ 4\ $+\ \A\C}のほうが\\&
\command{\ryuko\ +\
\A\C}よりダメージは大きい\\\hline\hline
\multicolumn{2}{|p{14.6cm}|}{
\PG{2}\ \ \ \DG{50}
}\\\bhline{2pt}
コンボ&
\command{近\C}\ \Cancel\ \command{$\rightarrow$\ \A}\ \Cancel\
\command{\tatsu\ +\ \A\ or\ \C} $\times\ 2$\ \\
&\SC\
\command{\ryuko\ +\ \A\ or\ \C}\ \Cancel\ \command{\hado\
$\times\ 4\ $+\ \A\C}\\\hline
補足&倒しきれるかどうかで\command{\hado\
$\times\ 4\ $+\ \A\C}を使うか決められるといい\\\bhline{2pt}
\end{tabular*}
\endgroup
\subsubsection{端コンボ}
\begingroup
 \renewcommand{\arraystretch}{1.2}
\begin{tabular*}{15.1cm}{@{\extracolsep{\fill}}|p{3em}||p{12.9cm}|}\hline
\multicolumn{2}{|p{14.6cm}|}{
\PG{1}\ \ \ \DG{0}
}\\\bhline{2pt}
コンボ&\command{近\C}\ \Cancel\ \command{$\rightarrow$\ \A}\ \Cancel\
\command{\tatsu\ +\ \A} $\times\ 3$\\
& \migi\ \command{\gyakuyoga\ +\ \B\D}\\\hline
補足&画面端であればダウン追撃が可能。\command{\tatsu\ +\ \C} $\times\ 3$後は不可\\\bhline{2pt}
\end{tabular*}
\endgroup
\newpage
\subsection{ネスツ京}
\subsubsection{中央コンボ}
\begingroup
 \renewcommand{\arraystretch}{1.2}
\begin{tabular*}{15.1cm}{@{\extracolsep{\fill}}|p{3em}||p{12.9cm}|}\hline
\multicolumn{2}{|p{14.6cm}|}{
\PG{0}\ \ \ \DG{0}
}\\\bhline{2pt}
コンボ&
\command{近\C}\ \Cancel\ \command{$\rightarrow$\ \B}\ \Cancel\
\command{\gyakuyoga\ +\ \B\ or\ \D}
%%コンボルート
\\\hline
補足&
ヒット確認簡単で\command{$\rightarrow$\ \B}まで入れ込みがほぼノーリスク。コンボのダメージも高い。強制ダウン。
%%補足
\\\bhline{2pt}%%\hline\hline
コンボ&
(\command{$\downarrow$\ \B}\ or\ \command{立\A}\ or\ \command{立\B}\ $\times\ 1\
\sim\ 2$)\ \Cancel\ \command{立\B}\\
& \Cancel\ \command{\hado\ +\ \A}\ \Cancel\ \command{\hado\
+\ \A}\ \Cancel\ \command{\B\ or\ \D}
%%コンボルート
\\\hline
補足&
小技からまとまったダメージがとれて横に押せる。
%%補足
\\\bhline{2pt}%%\hline\hline
コンボ&
\command{近\C}\ \Cancel\ \command{\hado\ +\ \D}\ \Cancel\
\command{B\ or\ \D}\\
& \migi\ \command{\gyakuyoga\ +\ \B\ or\ \D}
%%コンボルート
\\\hline
補足&
中央ノーゲージ最大。覚えなくてもいいかも。
%%補足
\\\hline\hline
\multicolumn{2}{|p{14.6cm}|}{
\PG{0}\ \ \ \DG{50}
}\\\bhline{2pt}
コンボ&
(\command{$\downarrow$\ \B}\ or\ \command{立\A}\ or\ \command{立\B}\ $\times\ 1\
\sim\ 2$)\ \Cancel\ \command{立\B}\\
& \Cancel\ \command{\hado\ +\ \A}\ \Cancel\
\command{\hado\ +\ \A}\\
& \DC\ \command{\gyakuyoga\ +\ \B\ or\ \D}
%%コンボルート
\\\hline
補足&
ダメージはそこまで伸びないので、倒しきれる時用。強制ダウン。
%%補足
\\\hline\hline
\multicolumn{2}{|p{14.6cm}|}{
\PG{1}\ \ \ \DG{0}
}\\\bhline{2pt}
コンボ&
\command{近\C}\ \Cancel\ \command{$\rightarrow$\ \B}\ \Cancel\ \command{\hado\
+\ \A\C}\\
& \Cancel\ \command{\gyakuyoga\ +\ \A\ or\ \C}\ \Cancel\
\command{$\rightarrow$\ +\ \C}\\
& \Cancel \command{\syoryu\ +\ \A\ or\ \C}
%%コンボルート
\\\hline
補足&
中央1ゲージの最大コンボ。
%%補足
\\\bhline{2pt}%%\hline\hline
コンボ&
(\command{$\downarrow$\ \B}\ or\ \command{立\A}\ or\ \command{立\B}\ $\times\ 1\
\sim\ 2$)\ \Cancel\ \command{立\B}\\
&\Cancel\ \command{\hado\hado\ +\ \A\ or\ \C}
%%コンボルート
\\\hline
補足&
小技始動1ゲージ最大。あまり使わない。
%%補足
\\\hline\hline
\multicolumn{2}{|p{14.6cm}|}{
\PG{1}\ \ \ \DG{50}
}\\\bhline{2pt}
コンボ&
\command{近\C}\ \Cancel\ \command{$\rightarrow$\ \B}\ \Cancel\ \command{\hado\
+\ \A\C}\ \Cancel\ \command{\hado\ +\ \A}\\
& \Cancel\ \command{\gyakuyoga\ +\ \A\ or\ \C}\ \Cancel\ \command{$\downarrow$\
+\ \A\ or\ \C}\\
& \DC\ \command{\gyakuyoga\ +\ \B\ or\ \D}
%%コンボルート
\\\hline
補足&強制ダウン。
%%補足
\\\bhline{2pt}%%\hline\hline
コンボ&
(\command{$\downarrow$\ \B}\ or\ \command{立\A}\ or\ \command{立\B}\ $\times\ 1\
\sim\ 2$)\ \Cancel\ \command{立\B}\\
& \Cancel\ \command{\hado\ +\ \A}\ \Cancel\
\command{\hado\ +\ \A}\\
& \DC\ \command{\orochi\ +\ \A\ or\ \C}
%%コンボルート
\\\hline
補足&
\command{\orochi\ +\ \A\ or\ \C}は少し遅目に入れた方がいい。
%%補足
\\\hline\hline
\end{tabular*}
\endgroup
\newpage
\subsubsection{中央コンボ続き}
\begingroup
 \renewcommand{\arraystretch}{1.2}
\begin{tabular*}{15.1cm}{@{\extracolsep{\fill}}|p{3em}||p{12.9cm}|}\hline
\multicolumn{2}{|p{14.6cm}|}{
\PG{2}\ \ \ \DG{0}
}\\\bhline{2pt}
コンボ&\command{近\C}\ \Cancel\ \command{$\rightarrow$\ \B}\ \Cancel\
\command{\orochi\ +\ \A\C}
%%コンボルート
\\\hline
補足&短い時間に大きなダメージ。
%%補足
\\\hline\hline
\multicolumn{2}{|p{14.6cm}|}{
\PG{2}\ \ \ \DG{50}
}\\\bhline{2pt}
コンボ&(\command{$\downarrow$\ \B}\ or\ \command{立\A}\ or\ \command{立\B}\ $\times\ 1\
\sim\ 2$)\ \Cancel\ \command{立\B}\\
& \Cancel\ \command{\hado\ +\ \A}\ \Cancel\
\command{\hado\ +\ \A}\\
& \DC\ \command{\orochi\ +\ \A\C}
%%コンボルート
\\\hline
補足&\command{\orochi\ +\ \A\C}はキャンセルが速すぎると\\
&\command{\hado\ +\
\A\C}に化けるので少し遅目にキャンセルする。
%%補足
\\\bhline{2pt}%%\hline\hline
\end{tabular*}
\endgroup
\newpage
\subsection{Mr.カラテ}
\subsubsection{中央コンボ}

\part{知っておくと得するかも知れないおまけ}
\section{ダメージ}
\begin{itemize}
  \item ヒット/ガード問わず相手に与えるダメージの最小値は4
  \item 必殺技以上をガードした時に受けるダメージ(削りダメージ)はヒット時のダメージの6分の1。ただし、一部削りダメージなしの例外あり。
  \item 一部のゲージ消費技の一部に固定ダメージと呼ばれる、補正の影響を受けないダメージが存在する(例:炎庵の豺華が固定90)
\end{itemize}
\end{document}
